\documentclass[11pt]{article}

    \usepackage[breakable]{tcolorbox}
    \usepackage{parskip} % Stop auto-indenting (to mimic markdown behaviour)
    
    \usepackage{iftex}
    \ifPDFTeX
    	\usepackage[T1]{fontenc}
    	\usepackage{mathpazo}
    \else
    	\usepackage{fontspec}
    \fi

    % Basic figure setup, for now with no caption control since it's done
    % automatically by Pandoc (which extracts ![](path) syntax from Markdown).
    \usepackage{graphicx}
    % Maintain compatibility with old templates. Remove in nbconvert 6.0
    \let\Oldincludegraphics\includegraphics
    % Ensure that by default, figures have no caption (until we provide a
    % proper Figure object with a Caption API and a way to capture that
    % in the conversion process - todo).
    \usepackage{caption}
    \DeclareCaptionFormat{nocaption}{}
    \captionsetup{format=nocaption,aboveskip=0pt,belowskip=0pt}

    \usepackage{float}
    \floatplacement{figure}{H} % forces figures to be placed at the correct location
    \usepackage{xcolor} % Allow colors to be defined
    \usepackage{enumerate} % Needed for markdown enumerations to work
    \usepackage{geometry} % Used to adjust the document margins
    \usepackage{amsmath} % Equations
    \usepackage{amssymb} % Equations
    \usepackage{textcomp} % defines textquotesingle
    % Hack from http://tex.stackexchange.com/a/47451/13684:
    \AtBeginDocument{%
        \def\PYZsq{\textquotesingle}% Upright quotes in Pygmentized code
    }
    \usepackage{upquote} % Upright quotes for verbatim code
    \usepackage{eurosym} % defines \euro
    \usepackage[mathletters]{ucs} % Extended unicode (utf-8) support
    \usepackage{fancyvrb} % verbatim replacement that allows latex
    \usepackage{grffile} % extends the file name processing of package graphics 
                         % to support a larger range
    \makeatletter % fix for old versions of grffile with XeLaTeX
    \@ifpackagelater{grffile}{2019/11/01}
    {
      % Do nothing on new versions
    }
    {
      \def\Gread@@xetex#1{%
        \IfFileExists{"\Gin@base".bb}%
        {\Gread@eps{\Gin@base.bb}}%
        {\Gread@@xetex@aux#1}%
      }
    }
    \makeatother
    \usepackage[Export]{adjustbox} % Used to constrain images to a maximum size
    \adjustboxset{max size={0.9\linewidth}{0.9\paperheight}}

    % The hyperref package gives us a pdf with properly built
    % internal navigation ('pdf bookmarks' for the table of contents,
    % internal cross-reference links, web links for URLs, etc.)
    \usepackage{hyperref}
    % The default LaTeX title has an obnoxious amount of whitespace. By default,
    % titling removes some of it. It also provides customization options.
    \usepackage{titling}
    \usepackage{longtable} % longtable support required by pandoc >1.10
    \usepackage{booktabs}  % table support for pandoc > 1.12.2
    \usepackage[inline]{enumitem} % IRkernel/repr support (it uses the enumerate* environment)
    \usepackage[normalem]{ulem} % ulem is needed to support strikethroughs (\sout)
                                % normalem makes italics be italics, not underlines
    \usepackage{mathrsfs}
    

    
    % Colors for the hyperref package
    \definecolor{urlcolor}{rgb}{0,.145,.698}
    \definecolor{linkcolor}{rgb}{.71,0.21,0.01}
    \definecolor{citecolor}{rgb}{.12,.54,.11}

    % ANSI colors
    \definecolor{ansi-black}{HTML}{3E424D}
    \definecolor{ansi-black-intense}{HTML}{282C36}
    \definecolor{ansi-red}{HTML}{E75C58}
    \definecolor{ansi-red-intense}{HTML}{B22B31}
    \definecolor{ansi-green}{HTML}{00A250}
    \definecolor{ansi-green-intense}{HTML}{007427}
    \definecolor{ansi-yellow}{HTML}{DDB62B}
    \definecolor{ansi-yellow-intense}{HTML}{B27D12}
    \definecolor{ansi-blue}{HTML}{208FFB}
    \definecolor{ansi-blue-intense}{HTML}{0065CA}
    \definecolor{ansi-magenta}{HTML}{D160C4}
    \definecolor{ansi-magenta-intense}{HTML}{A03196}
    \definecolor{ansi-cyan}{HTML}{60C6C8}
    \definecolor{ansi-cyan-intense}{HTML}{258F8F}
    \definecolor{ansi-white}{HTML}{C5C1B4}
    \definecolor{ansi-white-intense}{HTML}{A1A6B2}
    \definecolor{ansi-default-inverse-fg}{HTML}{FFFFFF}
    \definecolor{ansi-default-inverse-bg}{HTML}{000000}

    % common color for the border for error outputs.
    \definecolor{outerrorbackground}{HTML}{FFDFDF}

    % commands and environments needed by pandoc snippets
    % extracted from the output of `pandoc -s`
    \providecommand{\tightlist}{%
      \setlength{\itemsep}{0pt}\setlength{\parskip}{0pt}}
    \DefineVerbatimEnvironment{Highlighting}{Verbatim}{commandchars=\\\{\}}
    % Add ',fontsize=\small' for more characters per line
    \newenvironment{Shaded}{}{}
    \newcommand{\KeywordTok}[1]{\textcolor[rgb]{0.00,0.44,0.13}{\textbf{{#1}}}}
    \newcommand{\DataTypeTok}[1]{\textcolor[rgb]{0.56,0.13,0.00}{{#1}}}
    \newcommand{\DecValTok}[1]{\textcolor[rgb]{0.25,0.63,0.44}{{#1}}}
    \newcommand{\BaseNTok}[1]{\textcolor[rgb]{0.25,0.63,0.44}{{#1}}}
    \newcommand{\FloatTok}[1]{\textcolor[rgb]{0.25,0.63,0.44}{{#1}}}
    \newcommand{\CharTok}[1]{\textcolor[rgb]{0.25,0.44,0.63}{{#1}}}
    \newcommand{\StringTok}[1]{\textcolor[rgb]{0.25,0.44,0.63}{{#1}}}
    \newcommand{\CommentTok}[1]{\textcolor[rgb]{0.38,0.63,0.69}{\textit{{#1}}}}
    \newcommand{\OtherTok}[1]{\textcolor[rgb]{0.00,0.44,0.13}{{#1}}}
    \newcommand{\AlertTok}[1]{\textcolor[rgb]{1.00,0.00,0.00}{\textbf{{#1}}}}
    \newcommand{\FunctionTok}[1]{\textcolor[rgb]{0.02,0.16,0.49}{{#1}}}
    \newcommand{\RegionMarkerTok}[1]{{#1}}
    \newcommand{\ErrorTok}[1]{\textcolor[rgb]{1.00,0.00,0.00}{\textbf{{#1}}}}
    \newcommand{\NormalTok}[1]{{#1}}
    
    % Additional commands for more recent versions of Pandoc
    \newcommand{\ConstantTok}[1]{\textcolor[rgb]{0.53,0.00,0.00}{{#1}}}
    \newcommand{\SpecialCharTok}[1]{\textcolor[rgb]{0.25,0.44,0.63}{{#1}}}
    \newcommand{\VerbatimStringTok}[1]{\textcolor[rgb]{0.25,0.44,0.63}{{#1}}}
    \newcommand{\SpecialStringTok}[1]{\textcolor[rgb]{0.73,0.40,0.53}{{#1}}}
    \newcommand{\ImportTok}[1]{{#1}}
    \newcommand{\DocumentationTok}[1]{\textcolor[rgb]{0.73,0.13,0.13}{\textit{{#1}}}}
    \newcommand{\AnnotationTok}[1]{\textcolor[rgb]{0.38,0.63,0.69}{\textbf{\textit{{#1}}}}}
    \newcommand{\CommentVarTok}[1]{\textcolor[rgb]{0.38,0.63,0.69}{\textbf{\textit{{#1}}}}}
    \newcommand{\VariableTok}[1]{\textcolor[rgb]{0.10,0.09,0.49}{{#1}}}
    \newcommand{\ControlFlowTok}[1]{\textcolor[rgb]{0.00,0.44,0.13}{\textbf{{#1}}}}
    \newcommand{\OperatorTok}[1]{\textcolor[rgb]{0.40,0.40,0.40}{{#1}}}
    \newcommand{\BuiltInTok}[1]{{#1}}
    \newcommand{\ExtensionTok}[1]{{#1}}
    \newcommand{\PreprocessorTok}[1]{\textcolor[rgb]{0.74,0.48,0.00}{{#1}}}
    \newcommand{\AttributeTok}[1]{\textcolor[rgb]{0.49,0.56,0.16}{{#1}}}
    \newcommand{\InformationTok}[1]{\textcolor[rgb]{0.38,0.63,0.69}{\textbf{\textit{{#1}}}}}
    \newcommand{\WarningTok}[1]{\textcolor[rgb]{0.38,0.63,0.69}{\textbf{\textit{{#1}}}}}
    
    
    % Define a nice break command that doesn't care if a line doesn't already
    % exist.
    \def\br{\hspace*{\fill} \\* }
    % Math Jax compatibility definitions
    \def\gt{>}
    \def\lt{<}
    \let\Oldtex\TeX
    \let\Oldlatex\LaTeX
    \renewcommand{\TeX}{\textrm{\Oldtex}}
    \renewcommand{\LaTeX}{\textrm{\Oldlatex}}
    % Document parameters
    % Document title
    \title{hw0}
    
    
    
    
    
% Pygments definitions
\makeatletter
\def\PY@reset{\let\PY@it=\relax \let\PY@bf=\relax%
    \let\PY@ul=\relax \let\PY@tc=\relax%
    \let\PY@bc=\relax \let\PY@ff=\relax}
\def\PY@tok#1{\csname PY@tok@#1\endcsname}
\def\PY@toks#1+{\ifx\relax#1\empty\else%
    \PY@tok{#1}\expandafter\PY@toks\fi}
\def\PY@do#1{\PY@bc{\PY@tc{\PY@ul{%
    \PY@it{\PY@bf{\PY@ff{#1}}}}}}}
\def\PY#1#2{\PY@reset\PY@toks#1+\relax+\PY@do{#2}}

\expandafter\def\csname PY@tok@w\endcsname{\def\PY@tc##1{\textcolor[rgb]{0.73,0.73,0.73}{##1}}}
\expandafter\def\csname PY@tok@c\endcsname{\let\PY@it=\textit\def\PY@tc##1{\textcolor[rgb]{0.25,0.50,0.50}{##1}}}
\expandafter\def\csname PY@tok@cp\endcsname{\def\PY@tc##1{\textcolor[rgb]{0.74,0.48,0.00}{##1}}}
\expandafter\def\csname PY@tok@k\endcsname{\let\PY@bf=\textbf\def\PY@tc##1{\textcolor[rgb]{0.00,0.50,0.00}{##1}}}
\expandafter\def\csname PY@tok@kp\endcsname{\def\PY@tc##1{\textcolor[rgb]{0.00,0.50,0.00}{##1}}}
\expandafter\def\csname PY@tok@kt\endcsname{\def\PY@tc##1{\textcolor[rgb]{0.69,0.00,0.25}{##1}}}
\expandafter\def\csname PY@tok@o\endcsname{\def\PY@tc##1{\textcolor[rgb]{0.40,0.40,0.40}{##1}}}
\expandafter\def\csname PY@tok@ow\endcsname{\let\PY@bf=\textbf\def\PY@tc##1{\textcolor[rgb]{0.67,0.13,1.00}{##1}}}
\expandafter\def\csname PY@tok@nb\endcsname{\def\PY@tc##1{\textcolor[rgb]{0.00,0.50,0.00}{##1}}}
\expandafter\def\csname PY@tok@nf\endcsname{\def\PY@tc##1{\textcolor[rgb]{0.00,0.00,1.00}{##1}}}
\expandafter\def\csname PY@tok@nc\endcsname{\let\PY@bf=\textbf\def\PY@tc##1{\textcolor[rgb]{0.00,0.00,1.00}{##1}}}
\expandafter\def\csname PY@tok@nn\endcsname{\let\PY@bf=\textbf\def\PY@tc##1{\textcolor[rgb]{0.00,0.00,1.00}{##1}}}
\expandafter\def\csname PY@tok@ne\endcsname{\let\PY@bf=\textbf\def\PY@tc##1{\textcolor[rgb]{0.82,0.25,0.23}{##1}}}
\expandafter\def\csname PY@tok@nv\endcsname{\def\PY@tc##1{\textcolor[rgb]{0.10,0.09,0.49}{##1}}}
\expandafter\def\csname PY@tok@no\endcsname{\def\PY@tc##1{\textcolor[rgb]{0.53,0.00,0.00}{##1}}}
\expandafter\def\csname PY@tok@nl\endcsname{\def\PY@tc##1{\textcolor[rgb]{0.63,0.63,0.00}{##1}}}
\expandafter\def\csname PY@tok@ni\endcsname{\let\PY@bf=\textbf\def\PY@tc##1{\textcolor[rgb]{0.60,0.60,0.60}{##1}}}
\expandafter\def\csname PY@tok@na\endcsname{\def\PY@tc##1{\textcolor[rgb]{0.49,0.56,0.16}{##1}}}
\expandafter\def\csname PY@tok@nt\endcsname{\let\PY@bf=\textbf\def\PY@tc##1{\textcolor[rgb]{0.00,0.50,0.00}{##1}}}
\expandafter\def\csname PY@tok@nd\endcsname{\def\PY@tc##1{\textcolor[rgb]{0.67,0.13,1.00}{##1}}}
\expandafter\def\csname PY@tok@s\endcsname{\def\PY@tc##1{\textcolor[rgb]{0.73,0.13,0.13}{##1}}}
\expandafter\def\csname PY@tok@sd\endcsname{\let\PY@it=\textit\def\PY@tc##1{\textcolor[rgb]{0.73,0.13,0.13}{##1}}}
\expandafter\def\csname PY@tok@si\endcsname{\let\PY@bf=\textbf\def\PY@tc##1{\textcolor[rgb]{0.73,0.40,0.53}{##1}}}
\expandafter\def\csname PY@tok@se\endcsname{\let\PY@bf=\textbf\def\PY@tc##1{\textcolor[rgb]{0.73,0.40,0.13}{##1}}}
\expandafter\def\csname PY@tok@sr\endcsname{\def\PY@tc##1{\textcolor[rgb]{0.73,0.40,0.53}{##1}}}
\expandafter\def\csname PY@tok@ss\endcsname{\def\PY@tc##1{\textcolor[rgb]{0.10,0.09,0.49}{##1}}}
\expandafter\def\csname PY@tok@sx\endcsname{\def\PY@tc##1{\textcolor[rgb]{0.00,0.50,0.00}{##1}}}
\expandafter\def\csname PY@tok@m\endcsname{\def\PY@tc##1{\textcolor[rgb]{0.40,0.40,0.40}{##1}}}
\expandafter\def\csname PY@tok@gh\endcsname{\let\PY@bf=\textbf\def\PY@tc##1{\textcolor[rgb]{0.00,0.00,0.50}{##1}}}
\expandafter\def\csname PY@tok@gu\endcsname{\let\PY@bf=\textbf\def\PY@tc##1{\textcolor[rgb]{0.50,0.00,0.50}{##1}}}
\expandafter\def\csname PY@tok@gd\endcsname{\def\PY@tc##1{\textcolor[rgb]{0.63,0.00,0.00}{##1}}}
\expandafter\def\csname PY@tok@gi\endcsname{\def\PY@tc##1{\textcolor[rgb]{0.00,0.63,0.00}{##1}}}
\expandafter\def\csname PY@tok@gr\endcsname{\def\PY@tc##1{\textcolor[rgb]{1.00,0.00,0.00}{##1}}}
\expandafter\def\csname PY@tok@ge\endcsname{\let\PY@it=\textit}
\expandafter\def\csname PY@tok@gs\endcsname{\let\PY@bf=\textbf}
\expandafter\def\csname PY@tok@gp\endcsname{\let\PY@bf=\textbf\def\PY@tc##1{\textcolor[rgb]{0.00,0.00,0.50}{##1}}}
\expandafter\def\csname PY@tok@go\endcsname{\def\PY@tc##1{\textcolor[rgb]{0.53,0.53,0.53}{##1}}}
\expandafter\def\csname PY@tok@gt\endcsname{\def\PY@tc##1{\textcolor[rgb]{0.00,0.27,0.87}{##1}}}
\expandafter\def\csname PY@tok@err\endcsname{\def\PY@bc##1{\setlength{\fboxsep}{0pt}\fcolorbox[rgb]{1.00,0.00,0.00}{1,1,1}{\strut ##1}}}
\expandafter\def\csname PY@tok@kc\endcsname{\let\PY@bf=\textbf\def\PY@tc##1{\textcolor[rgb]{0.00,0.50,0.00}{##1}}}
\expandafter\def\csname PY@tok@kd\endcsname{\let\PY@bf=\textbf\def\PY@tc##1{\textcolor[rgb]{0.00,0.50,0.00}{##1}}}
\expandafter\def\csname PY@tok@kn\endcsname{\let\PY@bf=\textbf\def\PY@tc##1{\textcolor[rgb]{0.00,0.50,0.00}{##1}}}
\expandafter\def\csname PY@tok@kr\endcsname{\let\PY@bf=\textbf\def\PY@tc##1{\textcolor[rgb]{0.00,0.50,0.00}{##1}}}
\expandafter\def\csname PY@tok@bp\endcsname{\def\PY@tc##1{\textcolor[rgb]{0.00,0.50,0.00}{##1}}}
\expandafter\def\csname PY@tok@fm\endcsname{\def\PY@tc##1{\textcolor[rgb]{0.00,0.00,1.00}{##1}}}
\expandafter\def\csname PY@tok@vc\endcsname{\def\PY@tc##1{\textcolor[rgb]{0.10,0.09,0.49}{##1}}}
\expandafter\def\csname PY@tok@vg\endcsname{\def\PY@tc##1{\textcolor[rgb]{0.10,0.09,0.49}{##1}}}
\expandafter\def\csname PY@tok@vi\endcsname{\def\PY@tc##1{\textcolor[rgb]{0.10,0.09,0.49}{##1}}}
\expandafter\def\csname PY@tok@vm\endcsname{\def\PY@tc##1{\textcolor[rgb]{0.10,0.09,0.49}{##1}}}
\expandafter\def\csname PY@tok@sa\endcsname{\def\PY@tc##1{\textcolor[rgb]{0.73,0.13,0.13}{##1}}}
\expandafter\def\csname PY@tok@sb\endcsname{\def\PY@tc##1{\textcolor[rgb]{0.73,0.13,0.13}{##1}}}
\expandafter\def\csname PY@tok@sc\endcsname{\def\PY@tc##1{\textcolor[rgb]{0.73,0.13,0.13}{##1}}}
\expandafter\def\csname PY@tok@dl\endcsname{\def\PY@tc##1{\textcolor[rgb]{0.73,0.13,0.13}{##1}}}
\expandafter\def\csname PY@tok@s2\endcsname{\def\PY@tc##1{\textcolor[rgb]{0.73,0.13,0.13}{##1}}}
\expandafter\def\csname PY@tok@sh\endcsname{\def\PY@tc##1{\textcolor[rgb]{0.73,0.13,0.13}{##1}}}
\expandafter\def\csname PY@tok@s1\endcsname{\def\PY@tc##1{\textcolor[rgb]{0.73,0.13,0.13}{##1}}}
\expandafter\def\csname PY@tok@mb\endcsname{\def\PY@tc##1{\textcolor[rgb]{0.40,0.40,0.40}{##1}}}
\expandafter\def\csname PY@tok@mf\endcsname{\def\PY@tc##1{\textcolor[rgb]{0.40,0.40,0.40}{##1}}}
\expandafter\def\csname PY@tok@mh\endcsname{\def\PY@tc##1{\textcolor[rgb]{0.40,0.40,0.40}{##1}}}
\expandafter\def\csname PY@tok@mi\endcsname{\def\PY@tc##1{\textcolor[rgb]{0.40,0.40,0.40}{##1}}}
\expandafter\def\csname PY@tok@il\endcsname{\def\PY@tc##1{\textcolor[rgb]{0.40,0.40,0.40}{##1}}}
\expandafter\def\csname PY@tok@mo\endcsname{\def\PY@tc##1{\textcolor[rgb]{0.40,0.40,0.40}{##1}}}
\expandafter\def\csname PY@tok@ch\endcsname{\let\PY@it=\textit\def\PY@tc##1{\textcolor[rgb]{0.25,0.50,0.50}{##1}}}
\expandafter\def\csname PY@tok@cm\endcsname{\let\PY@it=\textit\def\PY@tc##1{\textcolor[rgb]{0.25,0.50,0.50}{##1}}}
\expandafter\def\csname PY@tok@cpf\endcsname{\let\PY@it=\textit\def\PY@tc##1{\textcolor[rgb]{0.25,0.50,0.50}{##1}}}
\expandafter\def\csname PY@tok@c1\endcsname{\let\PY@it=\textit\def\PY@tc##1{\textcolor[rgb]{0.25,0.50,0.50}{##1}}}
\expandafter\def\csname PY@tok@cs\endcsname{\let\PY@it=\textit\def\PY@tc##1{\textcolor[rgb]{0.25,0.50,0.50}{##1}}}

\def\PYZbs{\char`\\}
\def\PYZus{\char`\_}
\def\PYZob{\char`\{}
\def\PYZcb{\char`\}}
\def\PYZca{\char`\^}
\def\PYZam{\char`\&}
\def\PYZlt{\char`\<}
\def\PYZgt{\char`\>}
\def\PYZsh{\char`\#}
\def\PYZpc{\char`\%}
\def\PYZdl{\char`\$}
\def\PYZhy{\char`\-}
\def\PYZsq{\char`\'}
\def\PYZdq{\char`\"}
\def\PYZti{\char`\~}
% for compatibility with earlier versions
\def\PYZat{@}
\def\PYZlb{[}
\def\PYZrb{]}
\makeatother


    % For linebreaks inside Verbatim environment from package fancyvrb. 
    \makeatletter
        \newbox\Wrappedcontinuationbox 
        \newbox\Wrappedvisiblespacebox 
        \newcommand*\Wrappedvisiblespace {\textcolor{red}{\textvisiblespace}} 
        \newcommand*\Wrappedcontinuationsymbol {\textcolor{red}{\llap{\tiny$\m@th\hookrightarrow$}}} 
        \newcommand*\Wrappedcontinuationindent {3ex } 
        \newcommand*\Wrappedafterbreak {\kern\Wrappedcontinuationindent\copy\Wrappedcontinuationbox} 
        % Take advantage of the already applied Pygments mark-up to insert 
        % potential linebreaks for TeX processing. 
        %        {, <, #, %, $, ' and ": go to next line. 
        %        _, }, ^, &, >, - and ~: stay at end of broken line. 
        % Use of \textquotesingle for straight quote. 
        \newcommand*\Wrappedbreaksatspecials {% 
            \def\PYGZus{\discretionary{\char`\_}{\Wrappedafterbreak}{\char`\_}}% 
            \def\PYGZob{\discretionary{}{\Wrappedafterbreak\char`\{}{\char`\{}}% 
            \def\PYGZcb{\discretionary{\char`\}}{\Wrappedafterbreak}{\char`\}}}% 
            \def\PYGZca{\discretionary{\char`\^}{\Wrappedafterbreak}{\char`\^}}% 
            \def\PYGZam{\discretionary{\char`\&}{\Wrappedafterbreak}{\char`\&}}% 
            \def\PYGZlt{\discretionary{}{\Wrappedafterbreak\char`\<}{\char`\<}}% 
            \def\PYGZgt{\discretionary{\char`\>}{\Wrappedafterbreak}{\char`\>}}% 
            \def\PYGZsh{\discretionary{}{\Wrappedafterbreak\char`\#}{\char`\#}}% 
            \def\PYGZpc{\discretionary{}{\Wrappedafterbreak\char`\%}{\char`\%}}% 
            \def\PYGZdl{\discretionary{}{\Wrappedafterbreak\char`\$}{\char`\$}}% 
            \def\PYGZhy{\discretionary{\char`\-}{\Wrappedafterbreak}{\char`\-}}% 
            \def\PYGZsq{\discretionary{}{\Wrappedafterbreak\textquotesingle}{\textquotesingle}}% 
            \def\PYGZdq{\discretionary{}{\Wrappedafterbreak\char`\"}{\char`\"}}% 
            \def\PYGZti{\discretionary{\char`\~}{\Wrappedafterbreak}{\char`\~}}% 
        } 
        % Some characters . , ; ? ! / are not pygmentized. 
        % This macro makes them "active" and they will insert potential linebreaks 
        \newcommand*\Wrappedbreaksatpunct {% 
            \lccode`\~`\.\lowercase{\def~}{\discretionary{\hbox{\char`\.}}{\Wrappedafterbreak}{\hbox{\char`\.}}}% 
            \lccode`\~`\,\lowercase{\def~}{\discretionary{\hbox{\char`\,}}{\Wrappedafterbreak}{\hbox{\char`\,}}}% 
            \lccode`\~`\;\lowercase{\def~}{\discretionary{\hbox{\char`\;}}{\Wrappedafterbreak}{\hbox{\char`\;}}}% 
            \lccode`\~`\:\lowercase{\def~}{\discretionary{\hbox{\char`\:}}{\Wrappedafterbreak}{\hbox{\char`\:}}}% 
            \lccode`\~`\?\lowercase{\def~}{\discretionary{\hbox{\char`\?}}{\Wrappedafterbreak}{\hbox{\char`\?}}}% 
            \lccode`\~`\!\lowercase{\def~}{\discretionary{\hbox{\char`\!}}{\Wrappedafterbreak}{\hbox{\char`\!}}}% 
            \lccode`\~`\/\lowercase{\def~}{\discretionary{\hbox{\char`\/}}{\Wrappedafterbreak}{\hbox{\char`\/}}}% 
            \catcode`\.\active
            \catcode`\,\active 
            \catcode`\;\active
            \catcode`\:\active
            \catcode`\?\active
            \catcode`\!\active
            \catcode`\/\active 
            \lccode`\~`\~ 	
        }
    \makeatother

    \let\OriginalVerbatim=\Verbatim
    \makeatletter
    \renewcommand{\Verbatim}[1][1]{%
        %\parskip\z@skip
        \sbox\Wrappedcontinuationbox {\Wrappedcontinuationsymbol}%
        \sbox\Wrappedvisiblespacebox {\FV@SetupFont\Wrappedvisiblespace}%
        \def\FancyVerbFormatLine ##1{\hsize\linewidth
            \vtop{\raggedright\hyphenpenalty\z@\exhyphenpenalty\z@
                \doublehyphendemerits\z@\finalhyphendemerits\z@
                \strut ##1\strut}%
        }%
        % If the linebreak is at a space, the latter will be displayed as visible
        % space at end of first line, and a continuation symbol starts next line.
        % Stretch/shrink are however usually zero for typewriter font.
        \def\FV@Space {%
            \nobreak\hskip\z@ plus\fontdimen3\font minus\fontdimen4\font
            \discretionary{\copy\Wrappedvisiblespacebox}{\Wrappedafterbreak}
            {\kern\fontdimen2\font}%
        }%
        
        % Allow breaks at special characters using \PYG... macros.
        \Wrappedbreaksatspecials
        % Breaks at punctuation characters . , ; ? ! and / need catcode=\active 	
        \OriginalVerbatim[#1,codes*=\Wrappedbreaksatpunct]%
    }
    \makeatother

    % Exact colors from NB
    \definecolor{incolor}{HTML}{303F9F}
    \definecolor{outcolor}{HTML}{D84315}
    \definecolor{cellborder}{HTML}{CFCFCF}
    \definecolor{cellbackground}{HTML}{F7F7F7}
    
    % prompt
    \makeatletter
    \newcommand{\boxspacing}{\kern\kvtcb@left@rule\kern\kvtcb@boxsep}
    \makeatother
    \newcommand{\prompt}[4]{
        {\ttfamily\llap{{\color{#2}[#3]:\hspace{3pt}#4}}\vspace{-\baselineskip}}
    }
    

    
    % Prevent overflowing lines due to hard-to-break entities
    \sloppy 
    % Setup hyperref package
    \hypersetup{
      breaklinks=true,  % so long urls are correctly broken across lines
      colorlinks=true,
      urlcolor=urlcolor,
      linkcolor=linkcolor,
      citecolor=citecolor,
      }
    % Slightly bigger margins than the latex defaults
    
    \geometry{verbose,tmargin=1in,bmargin=1in,lmargin=1in,rmargin=1in}
    
    

\begin{document}
    
    \maketitle
    
    

    
    \hypertarget{homework-0}{%
\section{Homework 0}\label{homework-0}}

In this homework, we will go through basic linear algebra, NumPy, and
image manipulation using Python to get everyone on the same page for the
prerequisite skills for this class.

One of the aims of this homework assignment is to get you to start
getting comfortable searching for useful library functions online. So in
many of the functions you will implement, you will have to look up
helper functions.

    \begin{tcolorbox}[breakable, size=fbox, boxrule=1pt, pad at break*=1mm,colback=cellbackground, colframe=cellborder]
\prompt{In}{incolor}{12}{\boxspacing}
\begin{Verbatim}[commandchars=\\\{\}]
\PY{c+c1}{\PYZsh{} Imports the print function from newer versions of python}
\PY{k+kn}{from} \PY{n+nn}{\PYZus{}\PYZus{}future\PYZus{}\PYZus{}} \PY{k+kn}{import} \PY{n}{print\PYZus{}function}

\PY{c+c1}{\PYZsh{} Setup}

\PY{c+c1}{\PYZsh{} The Random module implements pseudo\PYZhy{}random number generators}
\PY{k+kn}{import} \PY{n+nn}{random} 

\PY{c+c1}{\PYZsh{} Numpy is the main package for scientific computing with Python. }
\PY{c+c1}{\PYZsh{} This will be one of our most used libraries in this class}
\PY{k+kn}{import} \PY{n+nn}{numpy} \PY{k}{as} \PY{n+nn}{np}

\PY{c+c1}{\PYZsh{} The Time library helps us time code runtimes}
\PY{k+kn}{import} \PY{n+nn}{time}


\PY{c+c1}{\PYZsh{} Imports all the methods in each of the files: linalg.py and imageManip.py}
\PY{k+kn}{from} \PY{n+nn}{linalg} \PY{k+kn}{import} \PY{o}{*}
\PY{k+kn}{from} \PY{n+nn}{imageManip} \PY{k+kn}{import} \PY{o}{*}


\PY{c+c1}{\PYZsh{} Matplotlib is a useful plotting library for python }
\PY{k+kn}{import} \PY{n+nn}{matplotlib}\PY{n+nn}{.}\PY{n+nn}{pyplot} \PY{k}{as} \PY{n+nn}{plt}
\PY{c+c1}{\PYZsh{} This code is to make matplotlib figures appear inline in the}
\PY{c+c1}{\PYZsh{} notebook rather than in a new window.}
\PY{o}{\PYZpc{}}\PY{k}{matplotlib} inline
\PY{n}{plt}\PY{o}{.}\PY{n}{rcParams}\PY{p}{[}\PY{l+s+s1}{\PYZsq{}}\PY{l+s+s1}{figure.figsize}\PY{l+s+s1}{\PYZsq{}}\PY{p}{]} \PY{o}{=} \PY{p}{(}\PY{l+m+mf}{10.0}\PY{p}{,} \PY{l+m+mf}{8.0}\PY{p}{)} \PY{c+c1}{\PYZsh{} set default size of plots}
\PY{n}{plt}\PY{o}{.}\PY{n}{rcParams}\PY{p}{[}\PY{l+s+s1}{\PYZsq{}}\PY{l+s+s1}{image.interpolation}\PY{l+s+s1}{\PYZsq{}}\PY{p}{]} \PY{o}{=} \PY{l+s+s1}{\PYZsq{}}\PY{l+s+s1}{nearest}\PY{l+s+s1}{\PYZsq{}}
\PY{n}{plt}\PY{o}{.}\PY{n}{rcParams}\PY{p}{[}\PY{l+s+s1}{\PYZsq{}}\PY{l+s+s1}{image.cmap}\PY{l+s+s1}{\PYZsq{}}\PY{p}{]} \PY{o}{=} \PY{l+s+s1}{\PYZsq{}}\PY{l+s+s1}{gray}\PY{l+s+s1}{\PYZsq{}}


\PY{c+c1}{\PYZsh{} Some more magic so that the notebook will reload external python modules;}
\PY{c+c1}{\PYZsh{} see http://stackoverflow.com/questions/1907993/autoreload\PYZhy{}of\PYZhy{}modules\PYZhy{}in\PYZhy{}ipython}
\PY{o}{\PYZpc{}}\PY{k}{load\PYZus{}ext} autoreload
\PY{o}{\PYZpc{}}\PY{k}{autoreload} 2
\PY{o}{\PYZpc{}}\PY{k}{reload\PYZus{}ext} autoreload
\end{Verbatim}
\end{tcolorbox}

    \begin{Verbatim}[commandchars=\\\{\}]
The autoreload extension is already loaded. To reload it, use:
  \%reload\_ext autoreload
    \end{Verbatim}

    \hypertarget{question-1-linear-algebra-and-numpy-review}{%
\section{Question 1: Linear Algebra and NumPy
Review}\label{question-1-linear-algebra-and-numpy-review}}

In this section, we will review linear algebra and learn how to use
vectors and matrices in python using numpy. By the end of this section,
you will have implemented all the required methods in
\texttt{linalg.py}.

    \hypertarget{question-1.1-5-points}{%
\subsection{Question 1.1 (5 points)}\label{question-1.1-5-points}}

First, let's test whether you can define the following matrices and
vectors using numpy. Look up \texttt{np.array()} for help. In the next
code block, define \(M\) as a \((4, 3)\) matrix, \(a\) as a \((1, 3)\)
row vector and \(b\) as a \((3, 1)\) column vector:

\[M = \begin{bmatrix}
1 & 2 & 3 \\
4 & 5 & 6 \\
7 & 8 & 9 \\
10 & 11 & 12 \end{bmatrix}
\]

\[a = \begin{bmatrix}
1 & 1 & 0
\end{bmatrix}
\]

\[b = \begin{bmatrix}
-1 \\ 2 \\ 5
\end{bmatrix}  
\]

    \begin{tcolorbox}[breakable, size=fbox, boxrule=1pt, pad at break*=1mm,colback=cellbackground, colframe=cellborder]
\prompt{In}{incolor}{13}{\boxspacing}
\begin{Verbatim}[commandchars=\\\{\}]
\PY{c+c1}{\PYZsh{}\PYZsh{}\PYZsh{} YOUR CODE HERE}
\PY{n}{M} \PY{o}{=} \PY{n}{np}\PY{o}{.}\PY{n}{array}\PY{p}{(}\PY{p}{[}\PY{p}{[}\PY{l+m+mi}{1}\PY{p}{,} \PY{l+m+mi}{2}\PY{p}{,} \PY{l+m+mi}{3}\PY{p}{]}\PY{p}{,} \PY{p}{[}\PY{l+m+mi}{4}\PY{p}{,} \PY{l+m+mi}{5}\PY{p}{,} \PY{l+m+mi}{6}\PY{p}{]}\PY{p}{,} \PY{p}{[}\PY{l+m+mi}{7}\PY{p}{,} \PY{l+m+mi}{8}\PY{p}{,} \PY{l+m+mi}{9}\PY{p}{]}\PY{p}{,} \PY{p}{[}\PY{l+m+mi}{10}\PY{p}{,} \PY{l+m+mi}{11}\PY{p}{,} \PY{l+m+mi}{12}\PY{p}{]}\PY{p}{]}\PY{p}{)}
\PY{n}{a} \PY{o}{=} \PY{n}{np}\PY{o}{.}\PY{n}{array}\PY{p}{(}\PY{p}{[}\PY{p}{[}\PY{l+m+mi}{1}\PY{p}{,} \PY{l+m+mi}{1}\PY{p}{,} \PY{l+m+mi}{0}\PY{p}{]}\PY{p}{]}\PY{p}{)}
\PY{n}{b} \PY{o}{=} \PY{n}{np}\PY{o}{.}\PY{n}{array}\PY{p}{(}\PY{p}{[}\PY{o}{\PYZhy{}}\PY{l+m+mi}{1}\PY{p}{,} \PY{l+m+mi}{2}\PY{p}{,} \PY{l+m+mi}{5}\PY{p}{]}\PY{p}{)}\PY{o}{.}\PY{n}{reshape}\PY{p}{(}\PY{l+m+mi}{3}\PY{p}{,} \PY{l+m+mi}{1}\PY{p}{)}
\PY{c+c1}{\PYZsh{}\PYZsh{}\PYZsh{} END CODE HERE}
\PY{n+nb}{print}\PY{p}{(}\PY{l+s+s2}{\PYZdq{}}\PY{l+s+s2}{M = }\PY{l+s+se}{\PYZbs{}n}\PY{l+s+s2}{\PYZdq{}}\PY{p}{,} \PY{n}{M}\PY{p}{)}
\PY{n+nb}{print}\PY{p}{(}\PY{l+s+s2}{\PYZdq{}}\PY{l+s+s2}{The size of M is: }\PY{l+s+s2}{\PYZdq{}}\PY{p}{,} \PY{n}{M}\PY{o}{.}\PY{n}{shape}\PY{p}{)}
\PY{n+nb}{print}\PY{p}{(}\PY{p}{)}
\PY{n+nb}{print}\PY{p}{(}\PY{l+s+s2}{\PYZdq{}}\PY{l+s+s2}{a = }\PY{l+s+s2}{\PYZdq{}}\PY{p}{,} \PY{n}{a}\PY{p}{)}
\PY{n+nb}{print}\PY{p}{(}\PY{l+s+s2}{\PYZdq{}}\PY{l+s+s2}{The size of a is: }\PY{l+s+s2}{\PYZdq{}}\PY{p}{,} \PY{n}{a}\PY{o}{.}\PY{n}{shape}\PY{p}{)}
\PY{n+nb}{print}\PY{p}{(}\PY{p}{)}
\PY{n+nb}{print}\PY{p}{(}\PY{l+s+s2}{\PYZdq{}}\PY{l+s+s2}{b = }\PY{l+s+s2}{\PYZdq{}}\PY{p}{,} \PY{n}{b}\PY{p}{)}
\PY{n+nb}{print}\PY{p}{(}\PY{l+s+s2}{\PYZdq{}}\PY{l+s+s2}{The size of b is: }\PY{l+s+s2}{\PYZdq{}}\PY{p}{,} \PY{n}{b}\PY{o}{.}\PY{n}{shape}\PY{p}{)}
\end{Verbatim}
\end{tcolorbox}

    \begin{Verbatim}[commandchars=\\\{\}]
M =
 [[ 1  2  3]
 [ 4  5  6]
 [ 7  8  9]
 [10 11 12]]
The size of M is:  (4, 3)

a =  [[1 1 0]]
The size of a is:  (1, 3)

b =  [[-1]
 [ 2]
 [ 5]]
The size of b is:  (3, 1)
    \end{Verbatim}

    \hypertarget{question-1.2-5-points}{%
\subsection{Question 1.2 (5 points)}\label{question-1.2-5-points}}

Implement the \texttt{dot\_product()} method in \texttt{linalg.py} and
check that it returns the correct answer for \(a^Tb\).

    \begin{tcolorbox}[breakable, size=fbox, boxrule=1pt, pad at break*=1mm,colback=cellbackground, colframe=cellborder]
\prompt{In}{incolor}{14}{\boxspacing}
\begin{Verbatim}[commandchars=\\\{\}]
\PY{c+c1}{\PYZsh{} Now, let\PYZsq{}s test out this dot product. Your answer should be [[1]].}
\PY{n}{aDotB} \PY{o}{=} \PY{n}{dot\PYZus{}product}\PY{p}{(}\PY{n}{a}\PY{p}{,} \PY{n}{b}\PY{p}{)}
\PY{n+nb}{print}\PY{p}{(}\PY{n}{aDotB}\PY{p}{)}

\PY{n+nb}{print}\PY{p}{(}\PY{l+s+s2}{\PYZdq{}}\PY{l+s+s2}{The size is: }\PY{l+s+s2}{\PYZdq{}}\PY{p}{,} \PY{n}{aDotB}\PY{o}{.}\PY{n}{shape}\PY{p}{)}
\end{Verbatim}
\end{tcolorbox}

    \begin{Verbatim}[commandchars=\\\{\}]
[[1]]
The size is:  (1, 1)
    \end{Verbatim}

    \hypertarget{question-1.3-5-points}{%
\subsection{Question 1.3 (5 points)}\label{question-1.3-5-points}}

Implement the \texttt{complicated\_matrix\_function()} method in
\texttt{linalg.py} and use it to compute \((ab)Ma^T\)

IMPORTANT NOTE: The \texttt{complicated\_matrix\_function()} method
expects all inputs to be two dimensional numpy arrays, as opposed to 1-D
arrays. This is an important distinction, because 2-D arrays can be
transposed, while 1-D arrays cannot.

To transpose a 2-D array, you can use the syntax \texttt{array.T}

    \begin{tcolorbox}[breakable, size=fbox, boxrule=1pt, pad at break*=1mm,colback=cellbackground, colframe=cellborder]
\prompt{In}{incolor}{15}{\boxspacing}
\begin{Verbatim}[commandchars=\\\{\}]
\PY{c+c1}{\PYZsh{} Your answer should be \PYZdl{}[[3], [9], [15], [21]]\PYZdl{} of shape(4, 1).}
\PY{n}{ans} \PY{o}{=} \PY{n}{complicated\PYZus{}matrix\PYZus{}function}\PY{p}{(}\PY{n}{M}\PY{p}{,} \PY{n}{a}\PY{p}{,} \PY{n}{b}\PY{p}{)}
\PY{n+nb}{print}\PY{p}{(}\PY{n}{ans}\PY{p}{)}
\PY{n+nb}{print}\PY{p}{(}\PY{p}{)}
\PY{n+nb}{print}\PY{p}{(}\PY{l+s+s2}{\PYZdq{}}\PY{l+s+s2}{The size is: }\PY{l+s+s2}{\PYZdq{}}\PY{p}{,} \PY{n}{ans}\PY{o}{.}\PY{n}{shape}\PY{p}{)}
\end{Verbatim}
\end{tcolorbox}

    \begin{Verbatim}[commandchars=\\\{\}]
[[ 3]
 [ 9]
 [15]
 [21]]

The size is:  (4, 1)
    \end{Verbatim}

    \begin{tcolorbox}[breakable, size=fbox, boxrule=1pt, pad at break*=1mm,colback=cellbackground, colframe=cellborder]
\prompt{In}{incolor}{16}{\boxspacing}
\begin{Verbatim}[commandchars=\\\{\}]
\PY{n}{M\PYZus{}2} \PY{o}{=} \PY{n}{np}\PY{o}{.}\PY{n}{array}\PY{p}{(}\PY{n+nb}{range}\PY{p}{(}\PY{l+m+mi}{4}\PY{p}{)}\PY{p}{)}\PY{o}{.}\PY{n}{reshape}\PY{p}{(}\PY{p}{(}\PY{l+m+mi}{2}\PY{p}{,}\PY{l+m+mi}{2}\PY{p}{)}\PY{p}{)}
\PY{n}{a\PYZus{}2} \PY{o}{=} \PY{n}{np}\PY{o}{.}\PY{n}{array}\PY{p}{(}\PY{p}{[}\PY{p}{[}\PY{l+m+mi}{1}\PY{p}{,}\PY{l+m+mi}{1}\PY{p}{]}\PY{p}{]}\PY{p}{)}
\PY{n}{b\PYZus{}2} \PY{o}{=} \PY{n}{np}\PY{o}{.}\PY{n}{array}\PY{p}{(}\PY{p}{[}\PY{p}{[}\PY{l+m+mi}{10}\PY{p}{,} \PY{l+m+mi}{10}\PY{p}{]}\PY{p}{]}\PY{p}{)}\PY{o}{.}\PY{n}{T}
\PY{n+nb}{print}\PY{p}{(}\PY{n}{M\PYZus{}2}\PY{o}{.}\PY{n}{shape}\PY{p}{)}
\PY{n+nb}{print}\PY{p}{(}\PY{n}{a\PYZus{}2}\PY{o}{.}\PY{n}{shape}\PY{p}{)}
\PY{n+nb}{print}\PY{p}{(}\PY{n}{b\PYZus{}2}\PY{o}{.}\PY{n}{shape}\PY{p}{)}
\PY{n+nb}{print}\PY{p}{(}\PY{p}{)}

\PY{c+c1}{\PYZsh{} Your answer should be \PYZdl{}[[20], [100]]\PYZdl{} of shape(2, 1).}
\PY{n}{ans} \PY{o}{=} \PY{n}{complicated\PYZus{}matrix\PYZus{}function}\PY{p}{(}\PY{n}{M\PYZus{}2}\PY{p}{,} \PY{n}{a\PYZus{}2}\PY{p}{,} \PY{n}{b\PYZus{}2}\PY{p}{)}
\PY{n+nb}{print}\PY{p}{(}\PY{n}{ans}\PY{p}{)}
\PY{n+nb}{print}\PY{p}{(}\PY{p}{)}
\PY{n+nb}{print}\PY{p}{(}\PY{l+s+s2}{\PYZdq{}}\PY{l+s+s2}{The size is: }\PY{l+s+s2}{\PYZdq{}}\PY{p}{,} \PY{n}{ans}\PY{o}{.}\PY{n}{shape}\PY{p}{)}
\end{Verbatim}
\end{tcolorbox}

    \begin{Verbatim}[commandchars=\\\{\}]
(2, 2)
(1, 2)
(2, 1)

[[ 20]
 [100]]

The size is:  (2, 1)
    \end{Verbatim}

    \hypertarget{question-1.4-10-points}{%
\subsection{Question 1.4 (10 points)}\label{question-1.4-10-points}}

Implement \texttt{eigen\_decomp()} and
\texttt{get\_eigen\_values\_and\_vectors()} methods. In this method,
perform eigenvalue decomposition on the following matrix and return the
largest k eigen values and corresponding eigen vectors (k is specified
in the method calls below).

\[M = \begin{bmatrix}
1 & 2 & 3 \\
4 & 5 & 6 \\
7 & 8 & 9 \end{bmatrix}
\]

    \begin{tcolorbox}[breakable, size=fbox, boxrule=1pt, pad at break*=1mm,colback=cellbackground, colframe=cellborder]
\prompt{In}{incolor}{34}{\boxspacing}
\begin{Verbatim}[commandchars=\\\{\}]
\PY{c+c1}{\PYZsh{} Let\PYZsq{}s define M.}
\PY{n}{M} \PY{o}{=} \PY{n}{np}\PY{o}{.}\PY{n}{array}\PY{p}{(}\PY{p}{[}\PY{p}{[}\PY{l+m+mi}{1}\PY{p}{,}\PY{l+m+mi}{2}\PY{p}{,}\PY{l+m+mi}{3}\PY{p}{]}\PY{p}{,}\PY{p}{[}\PY{l+m+mi}{4}\PY{p}{,}\PY{l+m+mi}{5}\PY{p}{,}\PY{l+m+mi}{6}\PY{p}{]}\PY{p}{,}\PY{p}{[}\PY{l+m+mi}{7}\PY{p}{,}\PY{l+m+mi}{8}\PY{p}{,}\PY{l+m+mi}{9}\PY{p}{]}\PY{p}{]}\PY{p}{)}
    
\PY{c+c1}{\PYZsh{} Now let\PYZsq{}s grab the first eigenvalue and first eigenvector.}
\PY{c+c1}{\PYZsh{} You should get back a single eigenvalue and a single eigenvector.}
\PY{n}{val}\PY{p}{,} \PY{n}{vec} \PY{o}{=} \PY{n}{get\PYZus{}eigen\PYZus{}values\PYZus{}and\PYZus{}vectors}\PY{p}{(}\PY{n}{M}\PY{p}{[}\PY{p}{:}\PY{p}{,}\PY{p}{:}\PY{l+m+mi}{3}\PY{p}{]}\PY{p}{,} \PY{l+m+mi}{1}\PY{p}{)}
\PY{n+nb}{print}\PY{p}{(}\PY{l+s+s2}{\PYZdq{}}\PY{l+s+s2}{First eigenvalue =}\PY{l+s+s2}{\PYZdq{}}\PY{p}{,} \PY{n}{val}\PY{p}{[}\PY{l+m+mi}{0}\PY{p}{]}\PY{p}{)}
\PY{n+nb}{print}\PY{p}{(}\PY{p}{)}
\PY{n+nb}{print}\PY{p}{(}\PY{l+s+s2}{\PYZdq{}}\PY{l+s+s2}{First eigenvector =}\PY{l+s+s2}{\PYZdq{}}\PY{p}{,} \PY{n}{vec}\PY{p}{[}\PY{l+m+mi}{0}\PY{p}{]}\PY{p}{)}
\PY{n+nb}{print}\PY{p}{(}\PY{p}{)}
\PY{k}{assert} \PY{n+nb}{len}\PY{p}{(}\PY{n}{vec}\PY{p}{)} \PY{o}{==} \PY{l+m+mi}{1}

\PY{c+c1}{\PYZsh{} Now, let\PYZsq{}s get the first two eigenvalues and eigenvectors.}
\PY{c+c1}{\PYZsh{} You should get back a list of two eigenvalues and a list of two eigenvector arrays.}
\PY{n}{val}\PY{p}{,} \PY{n}{vec} \PY{o}{=} \PY{n}{get\PYZus{}eigen\PYZus{}values\PYZus{}and\PYZus{}vectors}\PY{p}{(}\PY{n}{M}\PY{p}{[}\PY{p}{:}\PY{p}{,}\PY{p}{:}\PY{l+m+mi}{3}\PY{p}{]}\PY{p}{,} \PY{l+m+mi}{2}\PY{p}{)}
\PY{n+nb}{print}\PY{p}{(}\PY{l+s+s2}{\PYZdq{}}\PY{l+s+s2}{Eigenvalues =}\PY{l+s+s2}{\PYZdq{}}\PY{p}{,} \PY{n}{val}\PY{p}{)}
\PY{n+nb}{print}\PY{p}{(}\PY{p}{)}
\PY{n+nb}{print}\PY{p}{(}\PY{l+s+s2}{\PYZdq{}}\PY{l+s+s2}{Eigenvectors =}\PY{l+s+s2}{\PYZdq{}}\PY{p}{,} \PY{n}{vec}\PY{p}{)}
\PY{k}{assert} \PY{n+nb}{len}\PY{p}{(}\PY{n}{vec}\PY{p}{)} \PY{o}{==} \PY{l+m+mi}{2}
\end{Verbatim}
\end{tcolorbox}

    \begin{Verbatim}[commandchars=\\\{\}]
First eigenvalue = 16.116843969807043

First eigenvector = [-0.23197068724628617, -0.5253220933012336,
-0.8186734993561815]

Eigenvalues = [16.116843969807043, -1.1168439698070427]

Eigenvectors = [[-0.23197068724628617, -0.5253220933012336,
-0.8186734993561815], [-0.7858302387420671, -0.08675133925662833,
0.61232756022881]]
    \end{Verbatim}

    \hypertarget{question-1.5-10-points}{%
\subsection{Question 1.5 (10 points)}\label{question-1.5-10-points}}

To wrap up our overview of NumPy, let's implement something fun --- a
helper function for computing the Euclidean distance between two
\(n\)-dimensional points!

In the 2-dimensional case, computing the Euclidean distance reduces to
solving the Pythagorean theorem \(c = \sqrt{a^2 + b^2}\):

\begin{figure}
\centering
\includegraphics{pythagorean.png}
\caption{pythagorean.png}
\end{figure}

\ldots where, given two points \((x_1, y_1)\) and \((x_2, y_2)\),
\(a = x_1 - x_2\) and \(b = y_1 - y_2\).

More generally, given two \(n\)-dimensional vectors, the Euclidean
distance can be computed by:

\begin{enumerate}
\def\labelenumi{\arabic{enumi}.}
\tightlist
\item
  Performing an elementwise subtraction between the two vectors, to get
  \(n\) difference values.
\item
  Squaring each of the \(n\) difference values, and summing the squares.
\item
  Taking the square root of our sum.
\end{enumerate}

Alternatively, the Euclidean distance between length-\(n\) vectors \(u\)
and \(v\) can be written as:

\$ \quad\textbf{distance}(u, v) = \sqrt{\sum_{i=1}^n (u_i - v_i)^2} \$

Try implementing this function: first using native Python with a
\texttt{for} loop in the \texttt{euclidean\_distance\_native()}
function, then in NumPy \textbf{without any loops} in the
\texttt{euclidean\_distance\_numpy()} function. We've added some
\texttt{assert} statements here to help you check functionality (if it
prints nothing, then your implementation is correct)!

    \begin{tcolorbox}[breakable, size=fbox, boxrule=1pt, pad at break*=1mm,colback=cellbackground, colframe=cellborder]
\prompt{In}{incolor}{35}{\boxspacing}
\begin{Verbatim}[commandchars=\\\{\}]
\PY{c+c1}{\PYZsh{}\PYZsh{} Testing native Python function}
\PY{k}{assert} \PY{n}{euclidean\PYZus{}distance\PYZus{}native}\PY{p}{(}\PY{p}{[}\PY{l+m+mf}{7.0}\PY{p}{]}\PY{p}{,} \PY{p}{[}\PY{l+m+mf}{6.0}\PY{p}{]}\PY{p}{)} \PY{o}{==} \PY{l+m+mf}{1.0}
\PY{k}{assert} \PY{n}{euclidean\PYZus{}distance\PYZus{}native}\PY{p}{(}\PY{p}{[}\PY{l+m+mf}{7.0}\PY{p}{,} \PY{l+m+mf}{0.0}\PY{p}{]}\PY{p}{,} \PY{p}{[}\PY{l+m+mf}{3.0}\PY{p}{,} \PY{l+m+mf}{3.0}\PY{p}{]}\PY{p}{)} \PY{o}{==} \PY{l+m+mf}{5.0}
\PY{k}{assert} \PY{n}{euclidean\PYZus{}distance\PYZus{}native}\PY{p}{(}\PY{p}{[}\PY{l+m+mf}{7.0}\PY{p}{,} \PY{l+m+mf}{0.0}\PY{p}{,} \PY{l+m+mf}{0.0}\PY{p}{]}\PY{p}{,} \PY{p}{[}\PY{l+m+mf}{3.0}\PY{p}{,} \PY{l+m+mf}{0.0}\PY{p}{,} \PY{l+m+mf}{3.0}\PY{p}{]}\PY{p}{)} \PY{o}{==} \PY{l+m+mf}{5.0}
\end{Verbatim}
\end{tcolorbox}

    \begin{tcolorbox}[breakable, size=fbox, boxrule=1pt, pad at break*=1mm,colback=cellbackground, colframe=cellborder]
\prompt{In}{incolor}{36}{\boxspacing}
\begin{Verbatim}[commandchars=\\\{\}]
\PY{c+c1}{\PYZsh{}\PYZsh{} Testing NumPy function}
\PY{k}{assert} \PY{n}{euclidean\PYZus{}distance\PYZus{}numpy}\PY{p}{(}
    \PY{n}{np}\PY{o}{.}\PY{n}{array}\PY{p}{(}\PY{p}{[}\PY{l+m+mf}{7.0}\PY{p}{]}\PY{p}{)}\PY{p}{,}
    \PY{n}{np}\PY{o}{.}\PY{n}{array}\PY{p}{(}\PY{p}{[}\PY{l+m+mf}{6.0}\PY{p}{]}\PY{p}{)}
\PY{p}{)} \PY{o}{==} \PY{l+m+mf}{1.0}
\PY{k}{assert} \PY{n}{euclidean\PYZus{}distance\PYZus{}numpy}\PY{p}{(}
    \PY{n}{np}\PY{o}{.}\PY{n}{array}\PY{p}{(}\PY{p}{[}\PY{l+m+mf}{7.0}\PY{p}{,} \PY{l+m+mf}{0.0}\PY{p}{]}\PY{p}{)}\PY{p}{,}
    \PY{n}{np}\PY{o}{.}\PY{n}{array}\PY{p}{(}\PY{p}{[}\PY{l+m+mf}{3.0}\PY{p}{,} \PY{l+m+mf}{3.0}\PY{p}{]}\PY{p}{)}
\PY{p}{)} \PY{o}{==} \PY{l+m+mf}{5.0}
\PY{k}{assert} \PY{n}{euclidean\PYZus{}distance\PYZus{}numpy}\PY{p}{(}
    \PY{n}{np}\PY{o}{.}\PY{n}{array}\PY{p}{(}\PY{p}{[}\PY{l+m+mf}{7.0}\PY{p}{,} \PY{l+m+mf}{0.0}\PY{p}{,} \PY{l+m+mf}{0.0}\PY{p}{]}\PY{p}{)}\PY{p}{,}
    \PY{n}{np}\PY{o}{.}\PY{n}{array}\PY{p}{(}\PY{p}{[}\PY{l+m+mf}{3.0}\PY{p}{,} \PY{l+m+mf}{0.0}\PY{p}{,} \PY{l+m+mf}{3.0}\PY{p}{]}\PY{p}{)}
\PY{p}{)} \PY{o}{==} \PY{l+m+mf}{5.0}
\end{Verbatim}
\end{tcolorbox}

    Next, let's take a look at how these two implementations compare in
terms of runtime:

    \begin{tcolorbox}[breakable, size=fbox, boxrule=1pt, pad at break*=1mm,colback=cellbackground, colframe=cellborder]
\prompt{In}{incolor}{37}{\boxspacing}
\begin{Verbatim}[commandchars=\\\{\}]
\PY{n}{n} \PY{o}{=} \PY{l+m+mi}{1000}

\PY{c+c1}{\PYZsh{} Create some length\PYZhy{}n lists and/or n\PYZhy{}dimensional arrays}
\PY{n}{a} \PY{o}{=} \PY{p}{[}\PY{l+m+mf}{0.0}\PY{p}{]} \PY{o}{*} \PY{n}{n}
\PY{n}{b} \PY{o}{=} \PY{p}{[}\PY{l+m+mf}{10.0}\PY{p}{]} \PY{o}{*} \PY{n}{n}
\PY{n}{a\PYZus{}array} \PY{o}{=} \PY{n}{np}\PY{o}{.}\PY{n}{array}\PY{p}{(}\PY{n}{a}\PY{p}{)}
\PY{n}{b\PYZus{}array} \PY{o}{=} \PY{n}{np}\PY{o}{.}\PY{n}{array}\PY{p}{(}\PY{n}{b}\PY{p}{)}

\PY{c+c1}{\PYZsh{} Compute runtime for native implementation}
\PY{n}{start\PYZus{}time} \PY{o}{=} \PY{n}{time}\PY{o}{.}\PY{n}{time}\PY{p}{(}\PY{p}{)}
\PY{k}{for} \PY{n}{i} \PY{o+ow}{in} \PY{n+nb}{range}\PY{p}{(}\PY{l+m+mi}{10000}\PY{p}{)}\PY{p}{:}
    \PY{n}{euclidean\PYZus{}distance\PYZus{}native}\PY{p}{(}\PY{n}{a}\PY{p}{,} \PY{n}{b}\PY{p}{)}
\PY{n+nb}{print}\PY{p}{(}\PY{l+s+s2}{\PYZdq{}}\PY{l+s+s2}{Native:}\PY{l+s+s2}{\PYZdq{}}\PY{p}{,} \PY{p}{(}\PY{n}{time}\PY{o}{.}\PY{n}{time}\PY{p}{(}\PY{p}{)} \PY{o}{\PYZhy{}} \PY{n}{start\PYZus{}time}\PY{p}{)}\PY{p}{,} \PY{l+s+s2}{\PYZdq{}}\PY{l+s+s2}{seconds}\PY{l+s+s2}{\PYZdq{}}\PY{p}{)}

\PY{c+c1}{\PYZsh{} Compute runtime for numpy implementation}
\PY{c+c1}{\PYZsh{} Start by grabbing the current time in seconds}
\PY{n}{start\PYZus{}time} \PY{o}{=} \PY{n}{time}\PY{o}{.}\PY{n}{time}\PY{p}{(}\PY{p}{)}
\PY{k}{for} \PY{n}{i} \PY{o+ow}{in} \PY{n+nb}{range}\PY{p}{(}\PY{l+m+mi}{10000}\PY{p}{)}\PY{p}{:}
    \PY{n}{euclidean\PYZus{}distance\PYZus{}numpy}\PY{p}{(}\PY{n}{a\PYZus{}array}\PY{p}{,} \PY{n}{b\PYZus{}array}\PY{p}{)}
\PY{n+nb}{print}\PY{p}{(}\PY{l+s+s2}{\PYZdq{}}\PY{l+s+s2}{NumPy:}\PY{l+s+s2}{\PYZdq{}}\PY{p}{,} \PY{p}{(}\PY{n}{time}\PY{o}{.}\PY{n}{time}\PY{p}{(}\PY{p}{)} \PY{o}{\PYZhy{}} \PY{n}{start\PYZus{}time}\PY{p}{)}\PY{p}{,} \PY{l+s+s2}{\PYZdq{}}\PY{l+s+s2}{seconds}\PY{l+s+s2}{\PYZdq{}}\PY{p}{)}
\end{Verbatim}
\end{tcolorbox}

    \begin{Verbatim}[commandchars=\\\{\}]
Native: 2.227705240249634 seconds
NumPy: 0.14539384841918945 seconds
    \end{Verbatim}

    As you can see, doing vectorized calculations (i.e.~no for loops) with
NumPy results in significantly faster computations!

    \hypertarget{part-2-image-manipulation}{%
\section{Part 2: Image Manipulation}\label{part-2-image-manipulation}}

Now that you are familiar with using matrices and vectors. Let's load
some images and treat them as matrices and do some operations on them.
By the end of this section, you will have implemented all the methods in
\texttt{imageManip.py}

    \begin{tcolorbox}[breakable, size=fbox, boxrule=1pt, pad at break*=1mm,colback=cellbackground, colframe=cellborder]
\prompt{In}{incolor}{38}{\boxspacing}
\begin{Verbatim}[commandchars=\\\{\}]
\PY{c+c1}{\PYZsh{} Run this code to set the locations of the images we will be using. }
\PY{c+c1}{\PYZsh{} You can change these paths to point to your own images if you want to try them out for fun.}

\PY{n}{image1\PYZus{}path} \PY{o}{=} \PY{l+s+s1}{\PYZsq{}}\PY{l+s+s1}{./image1.jpg}\PY{l+s+s1}{\PYZsq{}}
\PY{n}{image2\PYZus{}path} \PY{o}{=} \PY{l+s+s1}{\PYZsq{}}\PY{l+s+s1}{./image2.jpg}\PY{l+s+s1}{\PYZsq{}}

\PY{k}{def} \PY{n+nf}{display}\PY{p}{(}\PY{n}{img}\PY{p}{)}\PY{p}{:}
    \PY{c+c1}{\PYZsh{} Show image}
    \PY{n}{plt}\PY{o}{.}\PY{n}{figure}\PY{p}{(}\PY{n}{figsize} \PY{o}{=} \PY{p}{(}\PY{l+m+mi}{5}\PY{p}{,}\PY{l+m+mi}{5}\PY{p}{)}\PY{p}{)}
    \PY{n}{plt}\PY{o}{.}\PY{n}{imshow}\PY{p}{(}\PY{n}{img}\PY{p}{)}
    \PY{n}{plt}\PY{o}{.}\PY{n}{axis}\PY{p}{(}\PY{l+s+s1}{\PYZsq{}}\PY{l+s+s1}{off}\PY{l+s+s1}{\PYZsq{}}\PY{p}{)}
    \PY{n}{plt}\PY{o}{.}\PY{n}{show}\PY{p}{(}\PY{p}{)}
\end{Verbatim}
\end{tcolorbox}

    \hypertarget{question-2.1-5-points}{%
\subsection{Question 2.1 (5 points)}\label{question-2.1-5-points}}

Implement the load method in imageManip.py and read the display method
below. We will use these two methods through the rest of the notebook to
visualize our work.

    \begin{tcolorbox}[breakable, size=fbox, boxrule=1pt, pad at break*=1mm,colback=cellbackground, colframe=cellborder]
\prompt{In}{incolor}{39}{\boxspacing}
\begin{Verbatim}[commandchars=\\\{\}]
\PY{n}{image1} \PY{o}{=} \PY{n}{load}\PY{p}{(}\PY{n}{image1\PYZus{}path}\PY{p}{)}

\PY{n}{display}\PY{p}{(}\PY{n}{image1}\PY{p}{)}
\end{Verbatim}
\end{tcolorbox}

    \begin{center}
    \adjustimage{max size={0.9\linewidth}{0.9\paperheight}}{output_21_0.png}
    \end{center}
    { \hspace*{\fill} \\}
    
    \hypertarget{question-2.2-5-points}{%
\subsection{Question 2.2 (5 points)}\label{question-2.2-5-points}}

One of the most common operations we perform when working with images is
rectangular \textbf{cropping}, or the action of removing unwanted outer
areas of an image.

Take a look at this code we've written to crop out everything but the
eyes of our baboon from above:

    \begin{tcolorbox}[breakable, size=fbox, boxrule=1pt, pad at break*=1mm,colback=cellbackground, colframe=cellborder]
\prompt{In}{incolor}{24}{\boxspacing}
\begin{Verbatim}[commandchars=\\\{\}]
\PY{n}{display}\PY{p}{(}\PY{n}{image1}\PY{p}{[}\PY{l+m+mi}{10}\PY{p}{:}\PY{l+m+mi}{60}\PY{p}{,} \PY{l+m+mi}{70}\PY{p}{:}\PY{l+m+mi}{230}\PY{p}{,} \PY{p}{:}\PY{p}{]}\PY{p}{)}
\end{Verbatim}
\end{tcolorbox}

    \begin{center}
    \adjustimage{max size={0.9\linewidth}{0.9\paperheight}}{output_23_0.png}
    \end{center}
    { \hspace*{\fill} \\}
    
    Implement the \texttt{crop\_image()} method by taking in the starting
row index, starting column index, number of rows, and number of columns,
and outputting the cropped image.

Then, in the cell below, see if you can pull out a 100x100 square from
each corner of the original \texttt{image1}: the top left, top right,
bottom left, and bottom right.

    \begin{tcolorbox}[breakable, size=fbox, boxrule=1pt, pad at break*=1mm,colback=cellbackground, colframe=cellborder]
\prompt{In}{incolor}{25}{\boxspacing}
\begin{Verbatim}[commandchars=\\\{\}]
\PY{n}{top\PYZus{}left\PYZus{}corner} \PY{o}{=} \PY{n}{crop\PYZus{}image}\PY{p}{(}\PY{n}{image1}\PY{p}{,} \PY{l+m+mi}{0}\PY{p}{,} \PY{l+m+mi}{0}\PY{p}{,} \PY{l+m+mi}{100}\PY{p}{,} \PY{l+m+mi}{100}\PY{p}{)} \PY{c+c1}{\PYZsh{}\PYZsh{} Replace \PYZdq{}None\PYZdq{} with a crop\PYZus{}image() function call.}
\PY{n}{top\PYZus{}right\PYZus{}corner} \PY{o}{=} \PY{n}{crop\PYZus{}image}\PY{p}{(}\PY{n}{image1}\PY{p}{,} \PY{l+m+mi}{0}\PY{p}{,} \PY{n}{image1}\PY{o}{.}\PY{n}{shape}\PY{p}{[}\PY{l+m+mi}{1}\PY{p}{]} \PY{o}{\PYZhy{}} \PY{l+m+mi}{100}\PY{p}{,} \PY{l+m+mi}{100}\PY{p}{,} \PY{l+m+mi}{100}\PY{p}{)} \PY{c+c1}{\PYZsh{}\PYZsh{} Replace \PYZdq{}None\PYZdq{} with a crop\PYZus{}image() function call.}
\PY{n}{bottom\PYZus{}left\PYZus{}corner} \PY{o}{=} \PY{n}{crop\PYZus{}image}\PY{p}{(}\PY{n}{image1}\PY{p}{,} \PY{n}{image1}\PY{o}{.}\PY{n}{shape}\PY{p}{[}\PY{l+m+mi}{0}\PY{p}{]} \PY{o}{\PYZhy{}} \PY{l+m+mi}{100}\PY{p}{,} \PY{l+m+mi}{0}\PY{p}{,} \PY{l+m+mi}{100}\PY{p}{,} \PY{l+m+mi}{100}\PY{p}{)} \PY{c+c1}{\PYZsh{}\PYZsh{} Replace \PYZdq{}None\PYZdq{} with a crop\PYZus{}image() function call.}
\PY{n}{bottom\PYZus{}right\PYZus{}corner} \PY{o}{=} \PY{n}{crop\PYZus{}image}\PY{p}{(}\PY{n}{image1}\PY{p}{,} \PY{n}{image1}\PY{o}{.}\PY{n}{shape}\PY{p}{[}\PY{l+m+mi}{0}\PY{p}{]} \PY{o}{\PYZhy{}} \PY{l+m+mi}{100}\PY{p}{,} \PY{n}{image1}\PY{o}{.}\PY{n}{shape}\PY{p}{[}\PY{l+m+mi}{1}\PY{p}{]} \PY{o}{\PYZhy{}} \PY{l+m+mi}{100}\PY{p}{,} \PY{l+m+mi}{100}\PY{p}{,} \PY{l+m+mi}{100}\PY{p}{)} \PY{c+c1}{\PYZsh{}\PYZsh{} Replace \PYZdq{}None\PYZdq{} with a crop\PYZus{}image() function call.}

\PY{n}{display}\PY{p}{(}\PY{n}{top\PYZus{}left\PYZus{}corner}\PY{p}{)}
\PY{n}{display}\PY{p}{(}\PY{n}{top\PYZus{}right\PYZus{}corner}\PY{p}{)}
\PY{n}{display}\PY{p}{(}\PY{n}{bottom\PYZus{}left\PYZus{}corner}\PY{p}{)}
\PY{n}{display}\PY{p}{(}\PY{n}{bottom\PYZus{}right\PYZus{}corner}\PY{p}{)}
\end{Verbatim}
\end{tcolorbox}

    \begin{center}
    \adjustimage{max size={0.9\linewidth}{0.9\paperheight}}{output_25_0.png}
    \end{center}
    { \hspace*{\fill} \\}
    
    \begin{center}
    \adjustimage{max size={0.9\linewidth}{0.9\paperheight}}{output_25_1.png}
    \end{center}
    { \hspace*{\fill} \\}
    
    \begin{center}
    \adjustimage{max size={0.9\linewidth}{0.9\paperheight}}{output_25_2.png}
    \end{center}
    { \hspace*{\fill} \\}
    
    \begin{center}
    \adjustimage{max size={0.9\linewidth}{0.9\paperheight}}{output_25_3.png}
    \end{center}
    { \hspace*{\fill} \\}
    
    \hypertarget{question-2.3-10-points}{%
\subsection{Question 2.3 (10 points)}\label{question-2.3-10-points}}

Implement the \texttt{dim\_image()} method by converting images
according to \(x_n = 0.5*x_p^2\) for every pixel, where \(x_n\) is the
new value and \(x_p\) is the original value.

Note: Since all the pixel values of the image are in the range
\([0, 1]\), the above formula will result in reducing these pixels
values and therefore make the image dimmer.

    \begin{tcolorbox}[breakable, size=fbox, boxrule=1pt, pad at break*=1mm,colback=cellbackground, colframe=cellborder]
\prompt{In}{incolor}{26}{\boxspacing}
\begin{Verbatim}[commandchars=\\\{\}]
\PY{n}{new\PYZus{}image} \PY{o}{=} \PY{n}{dim\PYZus{}image}\PY{p}{(}\PY{n}{image1}\PY{p}{)}
\PY{n}{display}\PY{p}{(}\PY{n}{new\PYZus{}image}\PY{p}{)}
\end{Verbatim}
\end{tcolorbox}

    \begin{center}
    \adjustimage{max size={0.9\linewidth}{0.9\paperheight}}{output_27_0.png}
    \end{center}
    { \hspace*{\fill} \\}
    
    \hypertarget{question-2.4-10-points}{%
\subsection{Question 2.4 (10 points)}\label{question-2.4-10-points}}

Let's try another commonly used operation: image resizing!

At a high level, image resizing should go something like this:

\begin{enumerate}
\def\labelenumi{\arabic{enumi}.}
\tightlist
\item
  We create an (initially empty) output array of the desired size,
  \texttt{output\_image}
\item
  We iterate over each pixel position \texttt{(i,j)} in the output image

  \begin{itemize}
  \tightlist
  \item
    For each output pixel, we compute a corresponding input pixel
    \texttt{(input\_i,\ input\_j)}
  \item
    We assign \texttt{output\_image{[}i,\ j,\ :{]}} to
    \texttt{input\_image{[}input\_i,\ input\_j,\ :{]}}
  \end{itemize}
\item
  We return the resized output image
\end{enumerate}

We want \texttt{input\_i} and \texttt{input\_j} to increase
proportionally with \texttt{i} and \texttt{j} respectively:

\begin{itemize}
\tightlist
\item
  \texttt{input\_i} can be computed as
  \texttt{int(i\ *\ row\_scale\_factor)}
\item
  \texttt{input\_j} can be computed as
  \texttt{int(j\ *\ col\_scale\_factor)}
\end{itemize}

\ldots where \texttt{int()} is a Python operation takes a float and
rounds it down to the nearest integer, and \texttt{row\_scale\_factor}
and \texttt{col\_scale\_factor} are constants computed from the image
input/output sizes.

Try to figure out what \texttt{row\_scale\_factor} and
\texttt{col\_scale\_factor} should be, then implement this algorithm in
the \texttt{resize\_image()} method! Then, run the cells below to test
out your image resizing algorithm!

When you downsize the baboon to 16x16, you should expect an output that
looks something like this:

\begin{figure}
\centering
\includegraphics{16_16_baboon.png}
\caption{16\_16\_baboon.png}
\end{figure}

When you stretch it horizontally to 50x400, you should get:

\begin{figure}
\centering
\includegraphics{50_400_baboon.png}
\caption{50\_400\_baboon.png}
\end{figure}

    \begin{tcolorbox}[breakable, size=fbox, boxrule=1pt, pad at break*=1mm,colback=cellbackground, colframe=cellborder]
\prompt{In}{incolor}{27}{\boxspacing}
\begin{Verbatim}[commandchars=\\\{\}]
\PY{n}{display}\PY{p}{(}\PY{n}{resize\PYZus{}image}\PY{p}{(}\PY{n}{image1}\PY{p}{,} \PY{l+m+mi}{16}\PY{p}{,} \PY{l+m+mi}{16}\PY{p}{)}\PY{p}{)}
\end{Verbatim}
\end{tcolorbox}

    \begin{center}
    \adjustimage{max size={0.9\linewidth}{0.9\paperheight}}{output_29_0.png}
    \end{center}
    { \hspace*{\fill} \\}
    
    \begin{tcolorbox}[breakable, size=fbox, boxrule=1pt, pad at break*=1mm,colback=cellbackground, colframe=cellborder]
\prompt{In}{incolor}{28}{\boxspacing}
\begin{Verbatim}[commandchars=\\\{\}]
\PY{n}{display}\PY{p}{(}\PY{n}{resize\PYZus{}image}\PY{p}{(}\PY{n}{image1}\PY{p}{,} \PY{l+m+mi}{50}\PY{p}{,} \PY{l+m+mi}{400}\PY{p}{)}\PY{p}{)}
\end{Verbatim}
\end{tcolorbox}

    \begin{center}
    \adjustimage{max size={0.9\linewidth}{0.9\paperheight}}{output_30_0.png}
    \end{center}
    { \hspace*{\fill} \\}
    
    \textbf{Question:} In the resize algorithm we describe above, the output
is populated by iterating over the indices of the output image. Could we
implement image resizing by iterating over the indices of the input
image instead? How do the two approaches compare?

\begin{quote}
Yes, but we would have to only iterate over the specific pixels we are
interested in. We could use xrange(0, input\_rows, input\_rows /
output\_rows) to skip over the values that are irrelevant. We would have
to initiate other variables outside to represent the current position in
the output image and increment accordingly.
\end{quote}

    \hypertarget{question-2.5-15-points}{%
\subsection{Question 2.5 (15 points)}\label{question-2.5-15-points}}

One more operation that you can try implementing is \textbf{image
rotation}. This is part of a real interview question that we've
encountered for actual computer vision jobs (notably at Facebook), and
we expect it to require quite a bit more thinking.

    \hypertarget{a-rotating-2d-coordinates-5-points}{%
\paragraph{a) Rotating 2D coordinates (5
points)}\label{a-rotating-2d-coordinates-5-points}}

Before we start thinking about rotating full images, let's start by
taking a look at rotating \texttt{(x,\ y)} coordinates:

\begin{figure}
\centering
\includegraphics{coordinate_rotation.png}
\caption{coordinate\_rotation.png}
\end{figure}

Using \texttt{np.cos()} and \texttt{np.sin()}, implement the
\texttt{rotate2d()} method to compute the coordinates \((x', y')\)
rotated by theta radians from \((x, y)\) using the lecture slides.

Once you've implemented the function, test your implementation below
using the assert statements (if it prints nothing, then your
implementation is correct):

    \begin{tcolorbox}[breakable, size=fbox, boxrule=1pt, pad at break*=1mm,colback=cellbackground, colframe=cellborder]
\prompt{In}{incolor}{29}{\boxspacing}
\begin{Verbatim}[commandchars=\\\{\}]
\PY{k}{assert} \PY{n}{rotate2d}\PY{p}{(}\PY{n}{np}\PY{o}{.}\PY{n}{array}\PY{p}{(}\PY{p}{[}\PY{l+m+mf}{1.0}\PY{p}{,} \PY{l+m+mf}{0.0}\PY{p}{]}\PY{p}{)}\PY{p}{,} \PY{l+m+mf}{0.0}\PY{p}{)}\PY{o}{.}\PY{n}{shape} \PY{o}{==} \PY{p}{(}
    \PY{l+m+mi}{2}\PY{p}{,}
\PY{p}{)}\PY{p}{,} \PY{l+s+s2}{\PYZdq{}}\PY{l+s+s2}{Output shape incorrect!}\PY{l+s+s2}{\PYZdq{}}
\PY{k}{assert} \PY{n}{np}\PY{o}{.}\PY{n}{allclose}\PY{p}{(}
    \PY{n}{rotate2d}\PY{p}{(}\PY{n}{np}\PY{o}{.}\PY{n}{array}\PY{p}{(}\PY{p}{[}\PY{l+m+mf}{1.0}\PY{p}{,} \PY{l+m+mf}{0.0}\PY{p}{]}\PY{p}{)}\PY{p}{,} \PY{l+m+mf}{0.0}\PY{p}{)}\PY{p}{,} \PY{n}{np}\PY{o}{.}\PY{n}{array}\PY{p}{(}\PY{p}{[}\PY{l+m+mf}{1.0}\PY{p}{,} \PY{l+m+mf}{0.0}\PY{p}{]}\PY{p}{)}
\PY{p}{)}\PY{p}{,} \PY{l+s+s2}{\PYZdq{}}\PY{l+s+s2}{\PYZdq{}}
\PY{k}{assert} \PY{n}{np}\PY{o}{.}\PY{n}{allclose}\PY{p}{(}
    \PY{n}{rotate2d}\PY{p}{(}\PY{n}{np}\PY{o}{.}\PY{n}{array}\PY{p}{(}\PY{p}{[}\PY{l+m+mf}{1.0}\PY{p}{,} \PY{l+m+mf}{0.0}\PY{p}{]}\PY{p}{)}\PY{p}{,} \PY{n}{np}\PY{o}{.}\PY{n}{pi} \PY{o}{/} \PY{l+m+mf}{2.0}\PY{p}{)}\PY{p}{,} \PY{n}{np}\PY{o}{.}\PY{n}{array}\PY{p}{(}\PY{p}{[}\PY{l+m+mf}{0.0}\PY{p}{,} \PY{l+m+mf}{1.0}\PY{p}{]}\PY{p}{)}
\PY{p}{)}\PY{p}{,} \PY{l+s+s2}{\PYZdq{}}\PY{l+s+s2}{\PYZdq{}}
\end{Verbatim}
\end{tcolorbox}

    Run the cell below to visualize a point as it's rotated around the
origin by a set of evenly-spaced angles! You should see 30 points
arranged in a circle.

    \begin{tcolorbox}[breakable, size=fbox, boxrule=1pt, pad at break*=1mm,colback=cellbackground, colframe=cellborder]
\prompt{In}{incolor}{30}{\boxspacing}
\begin{Verbatim}[commandchars=\\\{\}]
\PY{c+c1}{\PYZsh{} Visualize a point being rotated around the origin}
\PY{c+c1}{\PYZsh{} We\PYZsq{}ll use the matplotlib library for this!}
\PY{k+kn}{import} \PY{n+nn}{matplotlib}\PY{n+nn}{.}\PY{n+nn}{pyplot} \PY{k}{as} \PY{n+nn}{plt}

\PY{n}{points} \PY{o}{=} \PY{n}{np}\PY{o}{.}\PY{n}{zeros}\PY{p}{(}\PY{p}{(}\PY{l+m+mi}{30}\PY{p}{,} \PY{l+m+mi}{2}\PY{p}{)}\PY{p}{)}
\PY{k}{for} \PY{n}{i} \PY{o+ow}{in} \PY{n+nb}{range}\PY{p}{(}\PY{l+m+mi}{30}\PY{p}{)}\PY{p}{:}
    \PY{n}{points}\PY{p}{[}\PY{n}{i}\PY{p}{,} \PY{p}{:}\PY{p}{]} \PY{o}{=} \PY{n}{rotate2d}\PY{p}{(}\PY{n}{np}\PY{o}{.}\PY{n}{array}\PY{p}{(}\PY{p}{[}\PY{l+m+mf}{1.0}\PY{p}{,} \PY{l+m+mf}{0.0}\PY{p}{]}\PY{p}{)}\PY{p}{,} \PY{n}{i} \PY{o}{/} \PY{l+m+mf}{30.0} \PY{o}{*} \PY{p}{(}\PY{l+m+mi}{2} \PY{o}{*} \PY{n}{np}\PY{o}{.}\PY{n}{pi}\PY{p}{)}\PY{p}{)}

\PY{n}{plt}\PY{o}{.}\PY{n}{scatter}\PY{p}{(}\PY{n}{points}\PY{p}{[}\PY{p}{:}\PY{p}{,} \PY{l+m+mi}{0}\PY{p}{]}\PY{p}{,} \PY{n}{points}\PY{p}{[}\PY{p}{:}\PY{p}{,} \PY{l+m+mi}{1}\PY{p}{]}\PY{p}{)}
\PY{n}{plt}\PY{o}{.}\PY{n}{show}\PY{p}{(}\PY{p}{)}
\end{Verbatim}
\end{tcolorbox}

    \begin{center}
    \adjustimage{max size={0.9\linewidth}{0.9\paperheight}}{output_36_0.png}
    \end{center}
    { \hspace*{\fill} \\}
    
    \textbf{Question:} Our function currently only rotates input points
around the origin (0,0). Using the same \texttt{rotate2d} function, how
could we rotate the point around a center that wasn't at the origin?
\textbf{You'll need to do this when you implement image rotation below!}

\begin{quote}
Translate each pixel such that it is centered around the given point,
then perform the original rotation. When done, translate each pixel
back.
\end{quote}

    \hypertarget{b-rotate-image-10-points}{%
\paragraph{b) Rotate Image (10 points)}\label{b-rotate-image-10-points}}

Finally, use what you've learned about 2D rotations to create and
implement the \texttt{rotate\_image(input\_image,\ theta)} function!

For an input angle of \(\pi/4\) (45 degrees), the expected output is:

\begin{figure}
\centering
\includegraphics{rotated_output.png}
\caption{rotated\_output.png}
\end{figure}

\textbf{Hints:} - We recommend basing your code off your
\texttt{resize\_image()} implementation, and applying the same general
approach as before. Iterate over each pixel of an output image
\texttt{(i,\ j)}, then fill in a color from a corresponding input pixel
\texttt{(input\_i,\ input\_j)}. In this case, note that the output and
input images should be the same size. - If you run into an output pixel
whose corresponding input coordinates \texttt{input\_i} and
\texttt{input\_j} that are invalid, you can just ignore that pixel or
set it to black. - In our expected output above, we're rotating each
coordinate around the center of the image, not the origin. (the origin
is located at the top left)

    \begin{tcolorbox}[breakable, size=fbox, boxrule=1pt, pad at break*=1mm,colback=cellbackground, colframe=cellborder]
\prompt{In}{incolor}{31}{\boxspacing}
\begin{Verbatim}[commandchars=\\\{\}]
\PY{c+c1}{\PYZsh{}\PYZsh{} Test that your output matches the expected output}
\PY{n}{display}\PY{p}{(}\PY{n}{rotate\PYZus{}image}\PY{p}{(}\PY{n}{image1}\PY{p}{,} \PY{n}{np}\PY{o}{.}\PY{n}{pi} \PY{o}{/} \PY{l+m+mf}{4.0}\PY{p}{)}\PY{p}{)}
\end{Verbatim}
\end{tcolorbox}

    \begin{center}
    \adjustimage{max size={0.9\linewidth}{0.9\paperheight}}{output_39_0.png}
    \end{center}
    { \hspace*{\fill} \\}
    
    \begin{tcolorbox}[breakable, size=fbox, boxrule=1pt, pad at break*=1mm,colback=cellbackground, colframe=cellborder]
\prompt{In}{incolor}{ }{\boxspacing}
\begin{Verbatim}[commandchars=\\\{\}]

\end{Verbatim}
\end{tcolorbox}


    % Add a bibliography block to the postdoc
    
    
    
\end{document}
