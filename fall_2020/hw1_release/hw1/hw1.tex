\documentclass[11pt]{article}

    \usepackage[breakable]{tcolorbox}
    \usepackage{parskip} % Stop auto-indenting (to mimic markdown behaviour)
    
    \usepackage{iftex}
    \ifPDFTeX
    	\usepackage[T1]{fontenc}
    	\usepackage{mathpazo}
    \else
    	\usepackage{fontspec}
    \fi

    % Basic figure setup, for now with no caption control since it's done
    % automatically by Pandoc (which extracts ![](path) syntax from Markdown).
    \usepackage{graphicx}
    % Maintain compatibility with old templates. Remove in nbconvert 6.0
    \let\Oldincludegraphics\includegraphics
    % Ensure that by default, figures have no caption (until we provide a
    % proper Figure object with a Caption API and a way to capture that
    % in the conversion process - todo).
    \usepackage{caption}
    \DeclareCaptionFormat{nocaption}{}
    \captionsetup{format=nocaption,aboveskip=0pt,belowskip=0pt}

    \usepackage{float}
    \floatplacement{figure}{H} % forces figures to be placed at the correct location
    \usepackage{xcolor} % Allow colors to be defined
    \usepackage{enumerate} % Needed for markdown enumerations to work
    \usepackage{geometry} % Used to adjust the document margins
    \usepackage{amsmath} % Equations
    \usepackage{amssymb} % Equations
    \usepackage{textcomp} % defines textquotesingle
    % Hack from http://tex.stackexchange.com/a/47451/13684:
    \AtBeginDocument{%
        \def\PYZsq{\textquotesingle}% Upright quotes in Pygmentized code
    }
    \usepackage{upquote} % Upright quotes for verbatim code
    \usepackage{eurosym} % defines \euro
    \usepackage[mathletters]{ucs} % Extended unicode (utf-8) support
    \usepackage{fancyvrb} % verbatim replacement that allows latex
    \usepackage{grffile} % extends the file name processing of package graphics 
                         % to support a larger range
    \makeatletter % fix for old versions of grffile with XeLaTeX
    \@ifpackagelater{grffile}{2019/11/01}
    {
      % Do nothing on new versions
    }
    {
      \def\Gread@@xetex#1{%
        \IfFileExists{"\Gin@base".bb}%
        {\Gread@eps{\Gin@base.bb}}%
        {\Gread@@xetex@aux#1}%
      }
    }
    \makeatother
    \usepackage[Export]{adjustbox} % Used to constrain images to a maximum size
    \adjustboxset{max size={0.9\linewidth}{0.9\paperheight}}

    % The hyperref package gives us a pdf with properly built
    % internal navigation ('pdf bookmarks' for the table of contents,
    % internal cross-reference links, web links for URLs, etc.)
    \usepackage{hyperref}
    % The default LaTeX title has an obnoxious amount of whitespace. By default,
    % titling removes some of it. It also provides customization options.
    \usepackage{titling}
    \usepackage{longtable} % longtable support required by pandoc >1.10
    \usepackage{booktabs}  % table support for pandoc > 1.12.2
    \usepackage[inline]{enumitem} % IRkernel/repr support (it uses the enumerate* environment)
    \usepackage[normalem]{ulem} % ulem is needed to support strikethroughs (\sout)
                                % normalem makes italics be italics, not underlines
    \usepackage{mathrsfs}
    

    
    % Colors for the hyperref package
    \definecolor{urlcolor}{rgb}{0,.145,.698}
    \definecolor{linkcolor}{rgb}{.71,0.21,0.01}
    \definecolor{citecolor}{rgb}{.12,.54,.11}

    % ANSI colors
    \definecolor{ansi-black}{HTML}{3E424D}
    \definecolor{ansi-black-intense}{HTML}{282C36}
    \definecolor{ansi-red}{HTML}{E75C58}
    \definecolor{ansi-red-intense}{HTML}{B22B31}
    \definecolor{ansi-green}{HTML}{00A250}
    \definecolor{ansi-green-intense}{HTML}{007427}
    \definecolor{ansi-yellow}{HTML}{DDB62B}
    \definecolor{ansi-yellow-intense}{HTML}{B27D12}
    \definecolor{ansi-blue}{HTML}{208FFB}
    \definecolor{ansi-blue-intense}{HTML}{0065CA}
    \definecolor{ansi-magenta}{HTML}{D160C4}
    \definecolor{ansi-magenta-intense}{HTML}{A03196}
    \definecolor{ansi-cyan}{HTML}{60C6C8}
    \definecolor{ansi-cyan-intense}{HTML}{258F8F}
    \definecolor{ansi-white}{HTML}{C5C1B4}
    \definecolor{ansi-white-intense}{HTML}{A1A6B2}
    \definecolor{ansi-default-inverse-fg}{HTML}{FFFFFF}
    \definecolor{ansi-default-inverse-bg}{HTML}{000000}

    % common color for the border for error outputs.
    \definecolor{outerrorbackground}{HTML}{FFDFDF}

    % commands and environments needed by pandoc snippets
    % extracted from the output of `pandoc -s`
    \providecommand{\tightlist}{%
      \setlength{\itemsep}{0pt}\setlength{\parskip}{0pt}}
    \DefineVerbatimEnvironment{Highlighting}{Verbatim}{commandchars=\\\{\}}
    % Add ',fontsize=\small' for more characters per line
    \newenvironment{Shaded}{}{}
    \newcommand{\KeywordTok}[1]{\textcolor[rgb]{0.00,0.44,0.13}{\textbf{{#1}}}}
    \newcommand{\DataTypeTok}[1]{\textcolor[rgb]{0.56,0.13,0.00}{{#1}}}
    \newcommand{\DecValTok}[1]{\textcolor[rgb]{0.25,0.63,0.44}{{#1}}}
    \newcommand{\BaseNTok}[1]{\textcolor[rgb]{0.25,0.63,0.44}{{#1}}}
    \newcommand{\FloatTok}[1]{\textcolor[rgb]{0.25,0.63,0.44}{{#1}}}
    \newcommand{\CharTok}[1]{\textcolor[rgb]{0.25,0.44,0.63}{{#1}}}
    \newcommand{\StringTok}[1]{\textcolor[rgb]{0.25,0.44,0.63}{{#1}}}
    \newcommand{\CommentTok}[1]{\textcolor[rgb]{0.38,0.63,0.69}{\textit{{#1}}}}
    \newcommand{\OtherTok}[1]{\textcolor[rgb]{0.00,0.44,0.13}{{#1}}}
    \newcommand{\AlertTok}[1]{\textcolor[rgb]{1.00,0.00,0.00}{\textbf{{#1}}}}
    \newcommand{\FunctionTok}[1]{\textcolor[rgb]{0.02,0.16,0.49}{{#1}}}
    \newcommand{\RegionMarkerTok}[1]{{#1}}
    \newcommand{\ErrorTok}[1]{\textcolor[rgb]{1.00,0.00,0.00}{\textbf{{#1}}}}
    \newcommand{\NormalTok}[1]{{#1}}
    
    % Additional commands for more recent versions of Pandoc
    \newcommand{\ConstantTok}[1]{\textcolor[rgb]{0.53,0.00,0.00}{{#1}}}
    \newcommand{\SpecialCharTok}[1]{\textcolor[rgb]{0.25,0.44,0.63}{{#1}}}
    \newcommand{\VerbatimStringTok}[1]{\textcolor[rgb]{0.25,0.44,0.63}{{#1}}}
    \newcommand{\SpecialStringTok}[1]{\textcolor[rgb]{0.73,0.40,0.53}{{#1}}}
    \newcommand{\ImportTok}[1]{{#1}}
    \newcommand{\DocumentationTok}[1]{\textcolor[rgb]{0.73,0.13,0.13}{\textit{{#1}}}}
    \newcommand{\AnnotationTok}[1]{\textcolor[rgb]{0.38,0.63,0.69}{\textbf{\textit{{#1}}}}}
    \newcommand{\CommentVarTok}[1]{\textcolor[rgb]{0.38,0.63,0.69}{\textbf{\textit{{#1}}}}}
    \newcommand{\VariableTok}[1]{\textcolor[rgb]{0.10,0.09,0.49}{{#1}}}
    \newcommand{\ControlFlowTok}[1]{\textcolor[rgb]{0.00,0.44,0.13}{\textbf{{#1}}}}
    \newcommand{\OperatorTok}[1]{\textcolor[rgb]{0.40,0.40,0.40}{{#1}}}
    \newcommand{\BuiltInTok}[1]{{#1}}
    \newcommand{\ExtensionTok}[1]{{#1}}
    \newcommand{\PreprocessorTok}[1]{\textcolor[rgb]{0.74,0.48,0.00}{{#1}}}
    \newcommand{\AttributeTok}[1]{\textcolor[rgb]{0.49,0.56,0.16}{{#1}}}
    \newcommand{\InformationTok}[1]{\textcolor[rgb]{0.38,0.63,0.69}{\textbf{\textit{{#1}}}}}
    \newcommand{\WarningTok}[1]{\textcolor[rgb]{0.38,0.63,0.69}{\textbf{\textit{{#1}}}}}
    
    
    % Define a nice break command that doesn't care if a line doesn't already
    % exist.
    \def\br{\hspace*{\fill} \\* }
    % Math Jax compatibility definitions
    \def\gt{>}
    \def\lt{<}
    \let\Oldtex\TeX
    \let\Oldlatex\LaTeX
    \renewcommand{\TeX}{\textrm{\Oldtex}}
    \renewcommand{\LaTeX}{\textrm{\Oldlatex}}
    % Document parameters
    % Document title
    \title{hw1}
    
    
    
    
    
% Pygments definitions
\makeatletter
\def\PY@reset{\let\PY@it=\relax \let\PY@bf=\relax%
    \let\PY@ul=\relax \let\PY@tc=\relax%
    \let\PY@bc=\relax \let\PY@ff=\relax}
\def\PY@tok#1{\csname PY@tok@#1\endcsname}
\def\PY@toks#1+{\ifx\relax#1\empty\else%
    \PY@tok{#1}\expandafter\PY@toks\fi}
\def\PY@do#1{\PY@bc{\PY@tc{\PY@ul{%
    \PY@it{\PY@bf{\PY@ff{#1}}}}}}}
\def\PY#1#2{\PY@reset\PY@toks#1+\relax+\PY@do{#2}}

\expandafter\def\csname PY@tok@w\endcsname{\def\PY@tc##1{\textcolor[rgb]{0.73,0.73,0.73}{##1}}}
\expandafter\def\csname PY@tok@c\endcsname{\let\PY@it=\textit\def\PY@tc##1{\textcolor[rgb]{0.25,0.50,0.50}{##1}}}
\expandafter\def\csname PY@tok@cp\endcsname{\def\PY@tc##1{\textcolor[rgb]{0.74,0.48,0.00}{##1}}}
\expandafter\def\csname PY@tok@k\endcsname{\let\PY@bf=\textbf\def\PY@tc##1{\textcolor[rgb]{0.00,0.50,0.00}{##1}}}
\expandafter\def\csname PY@tok@kp\endcsname{\def\PY@tc##1{\textcolor[rgb]{0.00,0.50,0.00}{##1}}}
\expandafter\def\csname PY@tok@kt\endcsname{\def\PY@tc##1{\textcolor[rgb]{0.69,0.00,0.25}{##1}}}
\expandafter\def\csname PY@tok@o\endcsname{\def\PY@tc##1{\textcolor[rgb]{0.40,0.40,0.40}{##1}}}
\expandafter\def\csname PY@tok@ow\endcsname{\let\PY@bf=\textbf\def\PY@tc##1{\textcolor[rgb]{0.67,0.13,1.00}{##1}}}
\expandafter\def\csname PY@tok@nb\endcsname{\def\PY@tc##1{\textcolor[rgb]{0.00,0.50,0.00}{##1}}}
\expandafter\def\csname PY@tok@nf\endcsname{\def\PY@tc##1{\textcolor[rgb]{0.00,0.00,1.00}{##1}}}
\expandafter\def\csname PY@tok@nc\endcsname{\let\PY@bf=\textbf\def\PY@tc##1{\textcolor[rgb]{0.00,0.00,1.00}{##1}}}
\expandafter\def\csname PY@tok@nn\endcsname{\let\PY@bf=\textbf\def\PY@tc##1{\textcolor[rgb]{0.00,0.00,1.00}{##1}}}
\expandafter\def\csname PY@tok@ne\endcsname{\let\PY@bf=\textbf\def\PY@tc##1{\textcolor[rgb]{0.82,0.25,0.23}{##1}}}
\expandafter\def\csname PY@tok@nv\endcsname{\def\PY@tc##1{\textcolor[rgb]{0.10,0.09,0.49}{##1}}}
\expandafter\def\csname PY@tok@no\endcsname{\def\PY@tc##1{\textcolor[rgb]{0.53,0.00,0.00}{##1}}}
\expandafter\def\csname PY@tok@nl\endcsname{\def\PY@tc##1{\textcolor[rgb]{0.63,0.63,0.00}{##1}}}
\expandafter\def\csname PY@tok@ni\endcsname{\let\PY@bf=\textbf\def\PY@tc##1{\textcolor[rgb]{0.60,0.60,0.60}{##1}}}
\expandafter\def\csname PY@tok@na\endcsname{\def\PY@tc##1{\textcolor[rgb]{0.49,0.56,0.16}{##1}}}
\expandafter\def\csname PY@tok@nt\endcsname{\let\PY@bf=\textbf\def\PY@tc##1{\textcolor[rgb]{0.00,0.50,0.00}{##1}}}
\expandafter\def\csname PY@tok@nd\endcsname{\def\PY@tc##1{\textcolor[rgb]{0.67,0.13,1.00}{##1}}}
\expandafter\def\csname PY@tok@s\endcsname{\def\PY@tc##1{\textcolor[rgb]{0.73,0.13,0.13}{##1}}}
\expandafter\def\csname PY@tok@sd\endcsname{\let\PY@it=\textit\def\PY@tc##1{\textcolor[rgb]{0.73,0.13,0.13}{##1}}}
\expandafter\def\csname PY@tok@si\endcsname{\let\PY@bf=\textbf\def\PY@tc##1{\textcolor[rgb]{0.73,0.40,0.53}{##1}}}
\expandafter\def\csname PY@tok@se\endcsname{\let\PY@bf=\textbf\def\PY@tc##1{\textcolor[rgb]{0.73,0.40,0.13}{##1}}}
\expandafter\def\csname PY@tok@sr\endcsname{\def\PY@tc##1{\textcolor[rgb]{0.73,0.40,0.53}{##1}}}
\expandafter\def\csname PY@tok@ss\endcsname{\def\PY@tc##1{\textcolor[rgb]{0.10,0.09,0.49}{##1}}}
\expandafter\def\csname PY@tok@sx\endcsname{\def\PY@tc##1{\textcolor[rgb]{0.00,0.50,0.00}{##1}}}
\expandafter\def\csname PY@tok@m\endcsname{\def\PY@tc##1{\textcolor[rgb]{0.40,0.40,0.40}{##1}}}
\expandafter\def\csname PY@tok@gh\endcsname{\let\PY@bf=\textbf\def\PY@tc##1{\textcolor[rgb]{0.00,0.00,0.50}{##1}}}
\expandafter\def\csname PY@tok@gu\endcsname{\let\PY@bf=\textbf\def\PY@tc##1{\textcolor[rgb]{0.50,0.00,0.50}{##1}}}
\expandafter\def\csname PY@tok@gd\endcsname{\def\PY@tc##1{\textcolor[rgb]{0.63,0.00,0.00}{##1}}}
\expandafter\def\csname PY@tok@gi\endcsname{\def\PY@tc##1{\textcolor[rgb]{0.00,0.63,0.00}{##1}}}
\expandafter\def\csname PY@tok@gr\endcsname{\def\PY@tc##1{\textcolor[rgb]{1.00,0.00,0.00}{##1}}}
\expandafter\def\csname PY@tok@ge\endcsname{\let\PY@it=\textit}
\expandafter\def\csname PY@tok@gs\endcsname{\let\PY@bf=\textbf}
\expandafter\def\csname PY@tok@gp\endcsname{\let\PY@bf=\textbf\def\PY@tc##1{\textcolor[rgb]{0.00,0.00,0.50}{##1}}}
\expandafter\def\csname PY@tok@go\endcsname{\def\PY@tc##1{\textcolor[rgb]{0.53,0.53,0.53}{##1}}}
\expandafter\def\csname PY@tok@gt\endcsname{\def\PY@tc##1{\textcolor[rgb]{0.00,0.27,0.87}{##1}}}
\expandafter\def\csname PY@tok@err\endcsname{\def\PY@bc##1{\setlength{\fboxsep}{0pt}\fcolorbox[rgb]{1.00,0.00,0.00}{1,1,1}{\strut ##1}}}
\expandafter\def\csname PY@tok@kc\endcsname{\let\PY@bf=\textbf\def\PY@tc##1{\textcolor[rgb]{0.00,0.50,0.00}{##1}}}
\expandafter\def\csname PY@tok@kd\endcsname{\let\PY@bf=\textbf\def\PY@tc##1{\textcolor[rgb]{0.00,0.50,0.00}{##1}}}
\expandafter\def\csname PY@tok@kn\endcsname{\let\PY@bf=\textbf\def\PY@tc##1{\textcolor[rgb]{0.00,0.50,0.00}{##1}}}
\expandafter\def\csname PY@tok@kr\endcsname{\let\PY@bf=\textbf\def\PY@tc##1{\textcolor[rgb]{0.00,0.50,0.00}{##1}}}
\expandafter\def\csname PY@tok@bp\endcsname{\def\PY@tc##1{\textcolor[rgb]{0.00,0.50,0.00}{##1}}}
\expandafter\def\csname PY@tok@fm\endcsname{\def\PY@tc##1{\textcolor[rgb]{0.00,0.00,1.00}{##1}}}
\expandafter\def\csname PY@tok@vc\endcsname{\def\PY@tc##1{\textcolor[rgb]{0.10,0.09,0.49}{##1}}}
\expandafter\def\csname PY@tok@vg\endcsname{\def\PY@tc##1{\textcolor[rgb]{0.10,0.09,0.49}{##1}}}
\expandafter\def\csname PY@tok@vi\endcsname{\def\PY@tc##1{\textcolor[rgb]{0.10,0.09,0.49}{##1}}}
\expandafter\def\csname PY@tok@vm\endcsname{\def\PY@tc##1{\textcolor[rgb]{0.10,0.09,0.49}{##1}}}
\expandafter\def\csname PY@tok@sa\endcsname{\def\PY@tc##1{\textcolor[rgb]{0.73,0.13,0.13}{##1}}}
\expandafter\def\csname PY@tok@sb\endcsname{\def\PY@tc##1{\textcolor[rgb]{0.73,0.13,0.13}{##1}}}
\expandafter\def\csname PY@tok@sc\endcsname{\def\PY@tc##1{\textcolor[rgb]{0.73,0.13,0.13}{##1}}}
\expandafter\def\csname PY@tok@dl\endcsname{\def\PY@tc##1{\textcolor[rgb]{0.73,0.13,0.13}{##1}}}
\expandafter\def\csname PY@tok@s2\endcsname{\def\PY@tc##1{\textcolor[rgb]{0.73,0.13,0.13}{##1}}}
\expandafter\def\csname PY@tok@sh\endcsname{\def\PY@tc##1{\textcolor[rgb]{0.73,0.13,0.13}{##1}}}
\expandafter\def\csname PY@tok@s1\endcsname{\def\PY@tc##1{\textcolor[rgb]{0.73,0.13,0.13}{##1}}}
\expandafter\def\csname PY@tok@mb\endcsname{\def\PY@tc##1{\textcolor[rgb]{0.40,0.40,0.40}{##1}}}
\expandafter\def\csname PY@tok@mf\endcsname{\def\PY@tc##1{\textcolor[rgb]{0.40,0.40,0.40}{##1}}}
\expandafter\def\csname PY@tok@mh\endcsname{\def\PY@tc##1{\textcolor[rgb]{0.40,0.40,0.40}{##1}}}
\expandafter\def\csname PY@tok@mi\endcsname{\def\PY@tc##1{\textcolor[rgb]{0.40,0.40,0.40}{##1}}}
\expandafter\def\csname PY@tok@il\endcsname{\def\PY@tc##1{\textcolor[rgb]{0.40,0.40,0.40}{##1}}}
\expandafter\def\csname PY@tok@mo\endcsname{\def\PY@tc##1{\textcolor[rgb]{0.40,0.40,0.40}{##1}}}
\expandafter\def\csname PY@tok@ch\endcsname{\let\PY@it=\textit\def\PY@tc##1{\textcolor[rgb]{0.25,0.50,0.50}{##1}}}
\expandafter\def\csname PY@tok@cm\endcsname{\let\PY@it=\textit\def\PY@tc##1{\textcolor[rgb]{0.25,0.50,0.50}{##1}}}
\expandafter\def\csname PY@tok@cpf\endcsname{\let\PY@it=\textit\def\PY@tc##1{\textcolor[rgb]{0.25,0.50,0.50}{##1}}}
\expandafter\def\csname PY@tok@c1\endcsname{\let\PY@it=\textit\def\PY@tc##1{\textcolor[rgb]{0.25,0.50,0.50}{##1}}}
\expandafter\def\csname PY@tok@cs\endcsname{\let\PY@it=\textit\def\PY@tc##1{\textcolor[rgb]{0.25,0.50,0.50}{##1}}}

\def\PYZbs{\char`\\}
\def\PYZus{\char`\_}
\def\PYZob{\char`\{}
\def\PYZcb{\char`\}}
\def\PYZca{\char`\^}
\def\PYZam{\char`\&}
\def\PYZlt{\char`\<}
\def\PYZgt{\char`\>}
\def\PYZsh{\char`\#}
\def\PYZpc{\char`\%}
\def\PYZdl{\char`\$}
\def\PYZhy{\char`\-}
\def\PYZsq{\char`\'}
\def\PYZdq{\char`\"}
\def\PYZti{\char`\~}
% for compatibility with earlier versions
\def\PYZat{@}
\def\PYZlb{[}
\def\PYZrb{]}
\makeatother


    % For linebreaks inside Verbatim environment from package fancyvrb. 
    \makeatletter
        \newbox\Wrappedcontinuationbox 
        \newbox\Wrappedvisiblespacebox 
        \newcommand*\Wrappedvisiblespace {\textcolor{red}{\textvisiblespace}} 
        \newcommand*\Wrappedcontinuationsymbol {\textcolor{red}{\llap{\tiny$\m@th\hookrightarrow$}}} 
        \newcommand*\Wrappedcontinuationindent {3ex } 
        \newcommand*\Wrappedafterbreak {\kern\Wrappedcontinuationindent\copy\Wrappedcontinuationbox} 
        % Take advantage of the already applied Pygments mark-up to insert 
        % potential linebreaks for TeX processing. 
        %        {, <, #, %, $, ' and ": go to next line. 
        %        _, }, ^, &, >, - and ~: stay at end of broken line. 
        % Use of \textquotesingle for straight quote. 
        \newcommand*\Wrappedbreaksatspecials {% 
            \def\PYGZus{\discretionary{\char`\_}{\Wrappedafterbreak}{\char`\_}}% 
            \def\PYGZob{\discretionary{}{\Wrappedafterbreak\char`\{}{\char`\{}}% 
            \def\PYGZcb{\discretionary{\char`\}}{\Wrappedafterbreak}{\char`\}}}% 
            \def\PYGZca{\discretionary{\char`\^}{\Wrappedafterbreak}{\char`\^}}% 
            \def\PYGZam{\discretionary{\char`\&}{\Wrappedafterbreak}{\char`\&}}% 
            \def\PYGZlt{\discretionary{}{\Wrappedafterbreak\char`\<}{\char`\<}}% 
            \def\PYGZgt{\discretionary{\char`\>}{\Wrappedafterbreak}{\char`\>}}% 
            \def\PYGZsh{\discretionary{}{\Wrappedafterbreak\char`\#}{\char`\#}}% 
            \def\PYGZpc{\discretionary{}{\Wrappedafterbreak\char`\%}{\char`\%}}% 
            \def\PYGZdl{\discretionary{}{\Wrappedafterbreak\char`\$}{\char`\$}}% 
            \def\PYGZhy{\discretionary{\char`\-}{\Wrappedafterbreak}{\char`\-}}% 
            \def\PYGZsq{\discretionary{}{\Wrappedafterbreak\textquotesingle}{\textquotesingle}}% 
            \def\PYGZdq{\discretionary{}{\Wrappedafterbreak\char`\"}{\char`\"}}% 
            \def\PYGZti{\discretionary{\char`\~}{\Wrappedafterbreak}{\char`\~}}% 
        } 
        % Some characters . , ; ? ! / are not pygmentized. 
        % This macro makes them "active" and they will insert potential linebreaks 
        \newcommand*\Wrappedbreaksatpunct {% 
            \lccode`\~`\.\lowercase{\def~}{\discretionary{\hbox{\char`\.}}{\Wrappedafterbreak}{\hbox{\char`\.}}}% 
            \lccode`\~`\,\lowercase{\def~}{\discretionary{\hbox{\char`\,}}{\Wrappedafterbreak}{\hbox{\char`\,}}}% 
            \lccode`\~`\;\lowercase{\def~}{\discretionary{\hbox{\char`\;}}{\Wrappedafterbreak}{\hbox{\char`\;}}}% 
            \lccode`\~`\:\lowercase{\def~}{\discretionary{\hbox{\char`\:}}{\Wrappedafterbreak}{\hbox{\char`\:}}}% 
            \lccode`\~`\?\lowercase{\def~}{\discretionary{\hbox{\char`\?}}{\Wrappedafterbreak}{\hbox{\char`\?}}}% 
            \lccode`\~`\!\lowercase{\def~}{\discretionary{\hbox{\char`\!}}{\Wrappedafterbreak}{\hbox{\char`\!}}}% 
            \lccode`\~`\/\lowercase{\def~}{\discretionary{\hbox{\char`\/}}{\Wrappedafterbreak}{\hbox{\char`\/}}}% 
            \catcode`\.\active
            \catcode`\,\active 
            \catcode`\;\active
            \catcode`\:\active
            \catcode`\?\active
            \catcode`\!\active
            \catcode`\/\active 
            \lccode`\~`\~ 	
        }
    \makeatother

    \let\OriginalVerbatim=\Verbatim
    \makeatletter
    \renewcommand{\Verbatim}[1][1]{%
        %\parskip\z@skip
        \sbox\Wrappedcontinuationbox {\Wrappedcontinuationsymbol}%
        \sbox\Wrappedvisiblespacebox {\FV@SetupFont\Wrappedvisiblespace}%
        \def\FancyVerbFormatLine ##1{\hsize\linewidth
            \vtop{\raggedright\hyphenpenalty\z@\exhyphenpenalty\z@
                \doublehyphendemerits\z@\finalhyphendemerits\z@
                \strut ##1\strut}%
        }%
        % If the linebreak is at a space, the latter will be displayed as visible
        % space at end of first line, and a continuation symbol starts next line.
        % Stretch/shrink are however usually zero for typewriter font.
        \def\FV@Space {%
            \nobreak\hskip\z@ plus\fontdimen3\font minus\fontdimen4\font
            \discretionary{\copy\Wrappedvisiblespacebox}{\Wrappedafterbreak}
            {\kern\fontdimen2\font}%
        }%
        
        % Allow breaks at special characters using \PYG... macros.
        \Wrappedbreaksatspecials
        % Breaks at punctuation characters . , ; ? ! and / need catcode=\active 	
        \OriginalVerbatim[#1,codes*=\Wrappedbreaksatpunct]%
    }
    \makeatother

    % Exact colors from NB
    \definecolor{incolor}{HTML}{303F9F}
    \definecolor{outcolor}{HTML}{D84315}
    \definecolor{cellborder}{HTML}{CFCFCF}
    \definecolor{cellbackground}{HTML}{F7F7F7}
    
    % prompt
    \makeatletter
    \newcommand{\boxspacing}{\kern\kvtcb@left@rule\kern\kvtcb@boxsep}
    \makeatother
    \newcommand{\prompt}[4]{
        {\ttfamily\llap{{\color{#2}[#3]:\hspace{3pt}#4}}\vspace{-\baselineskip}}
    }
    

    
    % Prevent overflowing lines due to hard-to-break entities
    \sloppy 
    % Setup hyperref package
    \hypersetup{
      breaklinks=true,  % so long urls are correctly broken across lines
      colorlinks=true,
      urlcolor=urlcolor,
      linkcolor=linkcolor,
      citecolor=citecolor,
      }
    % Slightly bigger margins than the latex defaults
    
    \geometry{verbose,tmargin=1in,bmargin=1in,lmargin=1in,rmargin=1in}
    
    

\begin{document}
    
    \maketitle
    
    

    
    \hypertarget{homework-1}{%
\section{Homework 1}\label{homework-1}}

\emph{This notebook includes both coding and written questions. Please
hand in this notebook file with all the outputs and your answers to the
written questions.}

This assignment covers linear filters, convolution and correlation.

    \begin{tcolorbox}[breakable, size=fbox, boxrule=1pt, pad at break*=1mm,colback=cellbackground, colframe=cellborder]
\prompt{In}{incolor}{166}{\boxspacing}
\begin{Verbatim}[commandchars=\\\{\}]
\PY{c+c1}{\PYZsh{} Setup}
\PY{k+kn}{from} \PY{n+nn}{\PYZus{}\PYZus{}future\PYZus{}\PYZus{}} \PY{k+kn}{import} \PY{n}{print\PYZus{}function}
\PY{k+kn}{import} \PY{n+nn}{numpy} \PY{k}{as} \PY{n+nn}{np}
\PY{k+kn}{import} \PY{n+nn}{matplotlib}\PY{n+nn}{.}\PY{n+nn}{pyplot} \PY{k}{as} \PY{n+nn}{plt}
\PY{k+kn}{from} \PY{n+nn}{time} \PY{k+kn}{import} \PY{n}{time}
\PY{k+kn}{from} \PY{n+nn}{skimage} \PY{k+kn}{import} \PY{n}{io}


\PY{o}{\PYZpc{}}\PY{k}{matplotlib} inline
\PY{n}{plt}\PY{o}{.}\PY{n}{rcParams}\PY{p}{[}\PY{l+s+s1}{\PYZsq{}}\PY{l+s+s1}{figure.figsize}\PY{l+s+s1}{\PYZsq{}}\PY{p}{]} \PY{o}{=} \PY{p}{(}\PY{l+m+mf}{10.0}\PY{p}{,} \PY{l+m+mf}{8.0}\PY{p}{)} \PY{c+c1}{\PYZsh{} set default size of plots}
\PY{n}{plt}\PY{o}{.}\PY{n}{rcParams}\PY{p}{[}\PY{l+s+s1}{\PYZsq{}}\PY{l+s+s1}{image.interpolation}\PY{l+s+s1}{\PYZsq{}}\PY{p}{]} \PY{o}{=} \PY{l+s+s1}{\PYZsq{}}\PY{l+s+s1}{nearest}\PY{l+s+s1}{\PYZsq{}}
\PY{n}{plt}\PY{o}{.}\PY{n}{rcParams}\PY{p}{[}\PY{l+s+s1}{\PYZsq{}}\PY{l+s+s1}{image.cmap}\PY{l+s+s1}{\PYZsq{}}\PY{p}{]} \PY{o}{=} \PY{l+s+s1}{\PYZsq{}}\PY{l+s+s1}{gray}\PY{l+s+s1}{\PYZsq{}}

\PY{c+c1}{\PYZsh{} for auto\PYZhy{}reloading extenrnal modules}
\PY{o}{\PYZpc{}}\PY{k}{load\PYZus{}ext} autoreload
\PY{o}{\PYZpc{}}\PY{k}{autoreload} 2
\end{Verbatim}
\end{tcolorbox}

    \begin{Verbatim}[commandchars=\\\{\}]
The autoreload extension is already loaded. To reload it, use:
  \%reload\_ext autoreload
    \end{Verbatim}

    \hypertarget{part-1-convolutions}{%
\subsection{Part 1: Convolutions}\label{part-1-convolutions}}

\hypertarget{commutative-property-5-points}{%
\subsubsection{1.1 Commutative Property (5
points)}\label{commutative-property-5-points}}

Recall that the convolution of an image
\(f:\mathbb{R}^2\rightarrow \mathbb{R}\) and a kernel
\(h:\mathbb{R}^2\rightarrow\mathbb{R}\) is defined as follows:
\[(f*h)[m,n]=\sum_{i=-\infty}^\infty\sum_{j=-\infty}^\infty f[i,j]\cdot h[m-i,n-j]\]

Or equivalently, \begin{align}
(f*h)[m,n] &= \sum_{i=-\infty}^\infty\sum_{j=-\infty}^\infty h[i,j]\cdot f[m-i,n-j]\\
&= (h*f)[m,n]
\end{align}

Show that this is true (i.e.~prove that the convolution operator is
commutative: \(f*h = h*f\)).

    \textbf{Answer:} Let \(I = m - i\) and \(J = n - j\)

As \(i\) approaches \(\infty\), \(I\) approaches negative \(\infty\). As
\(i\) approaches negative \(\infty\), \(I\) approaches positive
\(\infty\). Similarly, as \(j\) approaches \(\infty\), \(J\) approaches
negative \(\infty\). As \(j\) approaches negative \(\infty\), \(J\)
approaches positive \(\infty\).

Therefore, the convolution
\((f * h)[m, n] = \sum_{i=-\infty}^{\infty}\sum_{j=-\infty}^{\infty}f[i, j] \cdot h[m - i, n - j]\)
can be rewritten as

\(\sum_{I=-\infty}^{\infty}\sum_{J=-\infty}^{\infty}f[m - I, n - J] \cdot h[I, J]\)

\$ =
\sum\emph{\{I=-\infty\}\^{}\{\infty\}\sum}\{J=-\infty\}\^{}\{\infty\}
h{[}I, J{]}\cdot f{[}m - I, n - J{]}\$

\$ =
\sum\emph{\{i=-\infty\}\^{}\{\infty\}\sum}\{j=-\infty\}\^{}\{\infty\}
h{[}i, j{]}\cdot f{[}m - i, n - j{]}\$

\$ = (h * f){[}m, n{]}\$

    \hypertarget{shift-invariance-5-points}{%
\subsubsection{1.2 Shift Invariance (5
points)}\label{shift-invariance-5-points}}

Let \(f\) be a function \(\mathbb{R}^2\rightarrow\mathbb{R}\). Consider
a system \(f\xrightarrow{s}g\), where \(g=(f*h)\) with some kernel
\(h:\mathbb{R}^2\rightarrow\mathbb{R}\). Also consider functions
\(f'[m,n] = f[m-m_0, n-n_0]\) and \(g'[m,n] = g[m-m_0, n-n_0]\).

Show that \(S\) defined by any kernel \(h\) is a shift invariant system
by showing that \(g' = (f'*h)\).

    \textbf{Answer:}

\$g'{[}m, n{]} = (f' * h){[}m, n{]} \$

\$ = (f * h){[}m - m\_0, n - n\_0{]} \$

\$ = \sum\emph{\{k=1\}\^{}\infty\sum}\{l=1\}\^{}\infty f{[}k, l{]}
\cdot h{[}m - m\_0 - i, n - n\_0 - j{]}\$

\$f' * h = (f * h){[}m - m\_0, n - n\_0{]} \$

\$ = \sum\emph{\{k=1\}\^{}\infty\sum}\{l=1\}\^{}\infty f{[}k, l{]}
\cdot h{[}m - m\_0 - i, n - n\_0 - j{]}\$

    \hypertarget{linearity-10-points}{%
\subsubsection{1.3 Linearity (10 points)}\label{linearity-10-points}}

Recall that a system S is considered a linear system if and only if it
satisfies the superposition property. In mathematical terms, a
(function) S is a linear invariant system iff it satisfies:

\[
S\{\alpha f_1[n,m] + \beta f_2[n,m]\} = \alpha S\{f_1[n,m]\} + \beta S\{f_2[n,m]\}
\]

Let \(f_1\) and \(f_2\) be functions
\(\mathbb{R}^2\rightarrow\mathbb{R}\). Consider a system
\(f\xrightarrow{s}g\), where \(g=(f*h)\) with some kernel
\(h:\mathbb{R}^2\rightarrow\mathbb{R}\).

Prove that \(S\) defined by any kernel \(h\) is linear by showing that
the superposition property holds.

    \textbf{Answer:}

Let \(f[n, m] = \alpha f_1[n, m] + \beta f_2[n, m]\).

\$S\{f{[}n, m{]}\} = S\{\alpha f\_1{[}n, m{]} + \beta f\_2{[}n, m{]}\}
\$

\$ =
\sum\emph{\{k=-\infty\}\^{}\{\infty\}\sum}\{l=-\infty\}\^{}\{\infty\}f{[}k,
l{]} \cdot h{[}n - k, m - 1{]}\$

\$ =
\sum\emph{\{k=-\infty\}\^{}\{\infty\}\sum}\{l=-\infty\}\^{}\{\infty\}(\alpha f\_1
+ \beta f\_2){[}k, l{]} \cdot h{[}n-k,m-1{]}\$

\$ =
\sum\emph{\{k=-\infty\}\^{}\{\infty\}\sum}\{l=-\infty\}\^{}\{\infty\}(\alpha f\_1{[}k,
l{]} + \beta f\_2{[}k, l{]}) \cdot h{[}n - k, m - 1{]}\$

\$ =
\sum\emph{\{k=-\infty\}\textsuperscript{\{\infty\}\sum\emph{\{l=-\infty\}\^{}\{\infty\}(\alpha f\_1{[}k,
l{]} \cdot h{[}n - k, m - 1{]}) +
\sum}\{k=-\infty\}}\{\infty\}\sum}\{l=-\infty\}\^{}\{\infty\}(\beta f\_2{[}k,
l{]}) \cdot h{[}n - k, m - 1{]})\$

\$ =
\alpha \sum\emph{\{k=-\infty\}\textsuperscript{\{\infty\}\sum\emph{\{l=-\infty\}\^{}\{\infty\}(f\_1{[}k,
l{]} \cdot h{[}n - k, m - 1{]}) +
\beta \sum}\{k=-\infty\}}\{\infty\}\sum}\{l=-\infty\}\^{}\{\infty\}(f\_2{[}k,
l{]}) \cdot h{[}n - k, m - 1{]})\$

\$ = \alpha S\{f\_1{[}n, m{]}\} + \beta S\{f\_w{[}n, m{]}\}\$

    \hypertarget{implementation-30-points}{%
\subsubsection{1.4 Implementation (30
points)}\label{implementation-30-points}}

In this section, you will implement two versions of convolution: -
\texttt{conv\_nested} - \texttt{conv\_fast}

First, run the code cell below to load the image to work with.

    \begin{tcolorbox}[breakable, size=fbox, boxrule=1pt, pad at break*=1mm,colback=cellbackground, colframe=cellborder]
\prompt{In}{incolor}{167}{\boxspacing}
\begin{Verbatim}[commandchars=\\\{\}]
\PY{c+c1}{\PYZsh{} Open image as grayscale}
\PY{n}{img} \PY{o}{=} \PY{n}{io}\PY{o}{.}\PY{n}{imread}\PY{p}{(}\PY{l+s+s1}{\PYZsq{}}\PY{l+s+s1}{dog.jpg}\PY{l+s+s1}{\PYZsq{}}\PY{p}{,} \PY{n}{as\PYZus{}gray}\PY{o}{=}\PY{k+kc}{True}\PY{p}{)}

\PY{c+c1}{\PYZsh{} Show image}
\PY{n}{plt}\PY{o}{.}\PY{n}{imshow}\PY{p}{(}\PY{n}{img}\PY{p}{)}
\PY{n}{plt}\PY{o}{.}\PY{n}{axis}\PY{p}{(}\PY{l+s+s1}{\PYZsq{}}\PY{l+s+s1}{off}\PY{l+s+s1}{\PYZsq{}}\PY{p}{)}
\PY{n}{plt}\PY{o}{.}\PY{n}{title}\PY{p}{(}\PY{l+s+s2}{\PYZdq{}}\PY{l+s+s2}{Isn}\PY{l+s+s2}{\PYZsq{}}\PY{l+s+s2}{t he cute?}\PY{l+s+s2}{\PYZdq{}}\PY{p}{)}
\PY{n}{plt}\PY{o}{.}\PY{n}{show}\PY{p}{(}\PY{p}{)}
\end{Verbatim}
\end{tcolorbox}

    \begin{center}
    \adjustimage{max size={0.9\linewidth}{0.9\paperheight}}{output_9_0.png}
    \end{center}
    { \hspace*{\fill} \\}
    
    Now, implement the function \textbf{\texttt{conv\_nested}} in
\textbf{\texttt{filters.py}}. This is a naive implementation of
convolution which uses 4 nested for-loops. It takes an image \(f\) and a
kernel \(h\) as inputs and outputs the convolved image \((f*h)\) that
has the same shape as the input image. This implementation should take a
few seconds to run.

\emph{- Hint: It may be easier to implement \((h*f)\)}

    We'll first test your \texttt{conv\_nested} function on a simple input.

    \begin{tcolorbox}[breakable, size=fbox, boxrule=1pt, pad at break*=1mm,colback=cellbackground, colframe=cellborder]
\prompt{In}{incolor}{168}{\boxspacing}
\begin{Verbatim}[commandchars=\\\{\}]
\PY{k+kn}{from} \PY{n+nn}{filters} \PY{k+kn}{import} \PY{n}{conv\PYZus{}nested}

\PY{c+c1}{\PYZsh{} Simple convolution kernel.}
\PY{n}{kernel} \PY{o}{=} \PY{n}{np}\PY{o}{.}\PY{n}{array}\PY{p}{(}
\PY{p}{[}
    \PY{p}{[}\PY{l+m+mi}{1}\PY{p}{,}\PY{l+m+mi}{0}\PY{p}{,}\PY{l+m+mi}{1}\PY{p}{]}\PY{p}{,}
    \PY{p}{[}\PY{l+m+mi}{0}\PY{p}{,}\PY{l+m+mi}{0}\PY{p}{,}\PY{l+m+mi}{0}\PY{p}{]}\PY{p}{,}
    \PY{p}{[}\PY{l+m+mi}{1}\PY{p}{,}\PY{l+m+mi}{0}\PY{p}{,}\PY{l+m+mi}{0}\PY{p}{]}
\PY{p}{]}\PY{p}{)}

\PY{c+c1}{\PYZsh{} Create a test image: a white square in the middle}
\PY{n}{test\PYZus{}img} \PY{o}{=} \PY{n}{np}\PY{o}{.}\PY{n}{zeros}\PY{p}{(}\PY{p}{(}\PY{l+m+mi}{9}\PY{p}{,} \PY{l+m+mi}{9}\PY{p}{)}\PY{p}{)}
\PY{n}{test\PYZus{}img}\PY{p}{[}\PY{l+m+mi}{3}\PY{p}{:}\PY{l+m+mi}{6}\PY{p}{,} \PY{l+m+mi}{3}\PY{p}{:}\PY{l+m+mi}{6}\PY{p}{]} \PY{o}{=} \PY{l+m+mi}{1}

\PY{c+c1}{\PYZsh{} Run your conv\PYZus{}nested function on the test image}
\PY{n}{test\PYZus{}output} \PY{o}{=} \PY{n}{conv\PYZus{}nested}\PY{p}{(}\PY{n}{test\PYZus{}img}\PY{p}{,} \PY{n}{kernel}\PY{p}{)}

\PY{c+c1}{\PYZsh{} Build the expected output}
\PY{n}{expected\PYZus{}output} \PY{o}{=} \PY{n}{np}\PY{o}{.}\PY{n}{zeros}\PY{p}{(}\PY{p}{(}\PY{l+m+mi}{9}\PY{p}{,} \PY{l+m+mi}{9}\PY{p}{)}\PY{p}{)}
\PY{n}{expected\PYZus{}output}\PY{p}{[}\PY{l+m+mi}{2}\PY{p}{:}\PY{l+m+mi}{7}\PY{p}{,} \PY{l+m+mi}{2}\PY{p}{:}\PY{l+m+mi}{7}\PY{p}{]} \PY{o}{=} \PY{l+m+mi}{1}
\PY{n}{expected\PYZus{}output}\PY{p}{[}\PY{l+m+mi}{5}\PY{p}{:}\PY{p}{,} \PY{l+m+mi}{5}\PY{p}{:}\PY{p}{]} \PY{o}{=} \PY{l+m+mi}{0}
\PY{n}{expected\PYZus{}output}\PY{p}{[}\PY{l+m+mi}{4}\PY{p}{,} \PY{l+m+mi}{2}\PY{p}{:}\PY{l+m+mi}{5}\PY{p}{]} \PY{o}{=} \PY{l+m+mi}{2}
\PY{n}{expected\PYZus{}output}\PY{p}{[}\PY{l+m+mi}{2}\PY{p}{:}\PY{l+m+mi}{5}\PY{p}{,} \PY{l+m+mi}{4}\PY{p}{]} \PY{o}{=} \PY{l+m+mi}{2}
\PY{n}{expected\PYZus{}output}\PY{p}{[}\PY{l+m+mi}{4}\PY{p}{,} \PY{l+m+mi}{4}\PY{p}{]} \PY{o}{=} \PY{l+m+mi}{3}

\PY{c+c1}{\PYZsh{} Plot the test image}
\PY{n}{plt}\PY{o}{.}\PY{n}{subplot}\PY{p}{(}\PY{l+m+mi}{1}\PY{p}{,}\PY{l+m+mi}{3}\PY{p}{,}\PY{l+m+mi}{1}\PY{p}{)}
\PY{n}{plt}\PY{o}{.}\PY{n}{imshow}\PY{p}{(}\PY{n}{test\PYZus{}img}\PY{p}{)}
\PY{n}{plt}\PY{o}{.}\PY{n}{title}\PY{p}{(}\PY{l+s+s1}{\PYZsq{}}\PY{l+s+s1}{Test image}\PY{l+s+s1}{\PYZsq{}}\PY{p}{)}
\PY{n}{plt}\PY{o}{.}\PY{n}{axis}\PY{p}{(}\PY{l+s+s1}{\PYZsq{}}\PY{l+s+s1}{off}\PY{l+s+s1}{\PYZsq{}}\PY{p}{)}

\PY{c+c1}{\PYZsh{} Plot your convolved image}
\PY{n}{plt}\PY{o}{.}\PY{n}{subplot}\PY{p}{(}\PY{l+m+mi}{1}\PY{p}{,}\PY{l+m+mi}{3}\PY{p}{,}\PY{l+m+mi}{2}\PY{p}{)}
\PY{n}{plt}\PY{o}{.}\PY{n}{imshow}\PY{p}{(}\PY{n}{test\PYZus{}output}\PY{p}{)}
\PY{n}{plt}\PY{o}{.}\PY{n}{title}\PY{p}{(}\PY{l+s+s1}{\PYZsq{}}\PY{l+s+s1}{Convolution}\PY{l+s+s1}{\PYZsq{}}\PY{p}{)}
\PY{n}{plt}\PY{o}{.}\PY{n}{axis}\PY{p}{(}\PY{l+s+s1}{\PYZsq{}}\PY{l+s+s1}{off}\PY{l+s+s1}{\PYZsq{}}\PY{p}{)}

\PY{c+c1}{\PYZsh{} Plot the exepected output}
\PY{n}{plt}\PY{o}{.}\PY{n}{subplot}\PY{p}{(}\PY{l+m+mi}{1}\PY{p}{,}\PY{l+m+mi}{3}\PY{p}{,}\PY{l+m+mi}{3}\PY{p}{)}
\PY{n}{plt}\PY{o}{.}\PY{n}{imshow}\PY{p}{(}\PY{n}{expected\PYZus{}output}\PY{p}{)}
\PY{n}{plt}\PY{o}{.}\PY{n}{title}\PY{p}{(}\PY{l+s+s1}{\PYZsq{}}\PY{l+s+s1}{Exepected output}\PY{l+s+s1}{\PYZsq{}}\PY{p}{)}
\PY{n}{plt}\PY{o}{.}\PY{n}{axis}\PY{p}{(}\PY{l+s+s1}{\PYZsq{}}\PY{l+s+s1}{off}\PY{l+s+s1}{\PYZsq{}}\PY{p}{)}
\PY{n}{plt}\PY{o}{.}\PY{n}{show}\PY{p}{(}\PY{p}{)}

\PY{c+c1}{\PYZsh{} Test if the output matches expected output}
\PY{k}{assert} \PY{n}{np}\PY{o}{.}\PY{n}{max}\PY{p}{(}\PY{n}{test\PYZus{}output} \PY{o}{\PYZhy{}} \PY{n}{expected\PYZus{}output}\PY{p}{)} \PY{o}{\PYZlt{}} \PY{l+m+mf}{1e\PYZhy{}10}\PY{p}{,} \PY{l+s+s2}{\PYZdq{}}\PY{l+s+s2}{Your solution is not correct.}\PY{l+s+s2}{\PYZdq{}}
\end{Verbatim}
\end{tcolorbox}

    \begin{center}
    \adjustimage{max size={0.9\linewidth}{0.9\paperheight}}{output_12_0.png}
    \end{center}
    { \hspace*{\fill} \\}
    
    Now let's test your \texttt{conv\_nested} function on a real image.

    \begin{tcolorbox}[breakable, size=fbox, boxrule=1pt, pad at break*=1mm,colback=cellbackground, colframe=cellborder]
\prompt{In}{incolor}{169}{\boxspacing}
\begin{Verbatim}[commandchars=\\\{\}]
\PY{k+kn}{from} \PY{n+nn}{filters} \PY{k+kn}{import} \PY{n}{conv\PYZus{}nested}

\PY{c+c1}{\PYZsh{} Simple convolution kernel.}
\PY{c+c1}{\PYZsh{} Feel free to change the kernel to see different outputs.}
\PY{n}{kernel} \PY{o}{=} \PY{n}{np}\PY{o}{.}\PY{n}{array}\PY{p}{(}
\PY{p}{[}
    \PY{p}{[}\PY{l+m+mi}{1}\PY{p}{,}\PY{l+m+mi}{0}\PY{p}{,}\PY{o}{\PYZhy{}}\PY{l+m+mi}{1}\PY{p}{]}\PY{p}{,}
    \PY{p}{[}\PY{l+m+mi}{2}\PY{p}{,}\PY{l+m+mi}{0}\PY{p}{,}\PY{o}{\PYZhy{}}\PY{l+m+mi}{2}\PY{p}{]}\PY{p}{,}
    \PY{p}{[}\PY{l+m+mi}{1}\PY{p}{,}\PY{l+m+mi}{0}\PY{p}{,}\PY{o}{\PYZhy{}}\PY{l+m+mi}{1}\PY{p}{]}
\PY{p}{]}\PY{p}{)}

\PY{n}{out} \PY{o}{=} \PY{n}{conv\PYZus{}nested}\PY{p}{(}\PY{n}{img}\PY{p}{,} \PY{n}{kernel}\PY{p}{)}

\PY{c+c1}{\PYZsh{} Plot original image}
\PY{n}{plt}\PY{o}{.}\PY{n}{subplot}\PY{p}{(}\PY{l+m+mi}{2}\PY{p}{,}\PY{l+m+mi}{2}\PY{p}{,}\PY{l+m+mi}{1}\PY{p}{)}
\PY{n}{plt}\PY{o}{.}\PY{n}{imshow}\PY{p}{(}\PY{n}{img}\PY{p}{)}
\PY{n}{plt}\PY{o}{.}\PY{n}{title}\PY{p}{(}\PY{l+s+s1}{\PYZsq{}}\PY{l+s+s1}{Original}\PY{l+s+s1}{\PYZsq{}}\PY{p}{)}
\PY{n}{plt}\PY{o}{.}\PY{n}{axis}\PY{p}{(}\PY{l+s+s1}{\PYZsq{}}\PY{l+s+s1}{off}\PY{l+s+s1}{\PYZsq{}}\PY{p}{)}

\PY{c+c1}{\PYZsh{} Plot your convolved image}
\PY{n}{plt}\PY{o}{.}\PY{n}{subplot}\PY{p}{(}\PY{l+m+mi}{2}\PY{p}{,}\PY{l+m+mi}{2}\PY{p}{,}\PY{l+m+mi}{3}\PY{p}{)}
\PY{n}{plt}\PY{o}{.}\PY{n}{imshow}\PY{p}{(}\PY{n}{out}\PY{p}{)}
\PY{n}{plt}\PY{o}{.}\PY{n}{title}\PY{p}{(}\PY{l+s+s1}{\PYZsq{}}\PY{l+s+s1}{Convolution}\PY{l+s+s1}{\PYZsq{}}\PY{p}{)}
\PY{n}{plt}\PY{o}{.}\PY{n}{axis}\PY{p}{(}\PY{l+s+s1}{\PYZsq{}}\PY{l+s+s1}{off}\PY{l+s+s1}{\PYZsq{}}\PY{p}{)}

\PY{c+c1}{\PYZsh{} Plot what you should get}
\PY{n}{solution\PYZus{}img} \PY{o}{=} \PY{n}{io}\PY{o}{.}\PY{n}{imread}\PY{p}{(}\PY{l+s+s1}{\PYZsq{}}\PY{l+s+s1}{convolved\PYZus{}dog.png}\PY{l+s+s1}{\PYZsq{}}\PY{p}{,} \PY{n}{as\PYZus{}gray}\PY{o}{=}\PY{k+kc}{True}\PY{p}{)}
\PY{n}{plt}\PY{o}{.}\PY{n}{subplot}\PY{p}{(}\PY{l+m+mi}{2}\PY{p}{,}\PY{l+m+mi}{2}\PY{p}{,}\PY{l+m+mi}{4}\PY{p}{)}
\PY{n}{plt}\PY{o}{.}\PY{n}{imshow}\PY{p}{(}\PY{n}{solution\PYZus{}img}\PY{p}{)}
\PY{n}{plt}\PY{o}{.}\PY{n}{title}\PY{p}{(}\PY{l+s+s1}{\PYZsq{}}\PY{l+s+s1}{What you should get}\PY{l+s+s1}{\PYZsq{}}\PY{p}{)}
\PY{n}{plt}\PY{o}{.}\PY{n}{axis}\PY{p}{(}\PY{l+s+s1}{\PYZsq{}}\PY{l+s+s1}{off}\PY{l+s+s1}{\PYZsq{}}\PY{p}{)}

\PY{n}{plt}\PY{o}{.}\PY{n}{show}\PY{p}{(}\PY{p}{)}
\end{Verbatim}
\end{tcolorbox}

    \begin{center}
    \adjustimage{max size={0.9\linewidth}{0.9\paperheight}}{output_14_0.png}
    \end{center}
    { \hspace*{\fill} \\}
    
    Let us implement a more efficient version of convolution using array
operations in numpy. As shown in the lecture, a convolution can be
considered as a sliding window that computes sum of the pixel values
weighted by the flipped kernel. The faster version will i) zero-pad an
image, ii) flip the kernel horizontally and vertically, and iii) compute
weighted sum of the neighborhood at each pixel.

First, implement the function \textbf{\texttt{zero\_pad}} in
\textbf{\texttt{filters.py}}.

    \begin{tcolorbox}[breakable, size=fbox, boxrule=1pt, pad at break*=1mm,colback=cellbackground, colframe=cellborder]
\prompt{In}{incolor}{170}{\boxspacing}
\begin{Verbatim}[commandchars=\\\{\}]
\PY{k+kn}{from} \PY{n+nn}{filters} \PY{k+kn}{import} \PY{n}{zero\PYZus{}pad}

\PY{n}{pad\PYZus{}width} \PY{o}{=} \PY{l+m+mi}{20} \PY{c+c1}{\PYZsh{} width of the padding on the left and right}
\PY{n}{pad\PYZus{}height} \PY{o}{=} \PY{l+m+mi}{40} \PY{c+c1}{\PYZsh{} height of the padding on the top and bottom}

\PY{n}{padded\PYZus{}img} \PY{o}{=} \PY{n}{zero\PYZus{}pad}\PY{p}{(}\PY{n}{img}\PY{p}{,} \PY{n}{pad\PYZus{}height}\PY{p}{,} \PY{n}{pad\PYZus{}width}\PY{p}{)}

\PY{c+c1}{\PYZsh{} Plot your padded dog}
\PY{n}{plt}\PY{o}{.}\PY{n}{subplot}\PY{p}{(}\PY{l+m+mi}{1}\PY{p}{,}\PY{l+m+mi}{2}\PY{p}{,}\PY{l+m+mi}{1}\PY{p}{)}
\PY{n}{plt}\PY{o}{.}\PY{n}{imshow}\PY{p}{(}\PY{n}{padded\PYZus{}img}\PY{p}{)}
\PY{n}{plt}\PY{o}{.}\PY{n}{title}\PY{p}{(}\PY{l+s+s1}{\PYZsq{}}\PY{l+s+s1}{Padded dog}\PY{l+s+s1}{\PYZsq{}}\PY{p}{)}
\PY{n}{plt}\PY{o}{.}\PY{n}{axis}\PY{p}{(}\PY{l+s+s1}{\PYZsq{}}\PY{l+s+s1}{off}\PY{l+s+s1}{\PYZsq{}}\PY{p}{)}

\PY{c+c1}{\PYZsh{} Plot what you should get}
\PY{n}{solution\PYZus{}img} \PY{o}{=} \PY{n}{io}\PY{o}{.}\PY{n}{imread}\PY{p}{(}\PY{l+s+s1}{\PYZsq{}}\PY{l+s+s1}{padded\PYZus{}dog.jpg}\PY{l+s+s1}{\PYZsq{}}\PY{p}{,} \PY{n}{as\PYZus{}gray}\PY{o}{=}\PY{k+kc}{True}\PY{p}{)}
\PY{n}{plt}\PY{o}{.}\PY{n}{subplot}\PY{p}{(}\PY{l+m+mi}{1}\PY{p}{,}\PY{l+m+mi}{2}\PY{p}{,}\PY{l+m+mi}{2}\PY{p}{)}
\PY{n}{plt}\PY{o}{.}\PY{n}{imshow}\PY{p}{(}\PY{n}{solution\PYZus{}img}\PY{p}{)}
\PY{n}{plt}\PY{o}{.}\PY{n}{title}\PY{p}{(}\PY{l+s+s1}{\PYZsq{}}\PY{l+s+s1}{What you should get}\PY{l+s+s1}{\PYZsq{}}\PY{p}{)}
\PY{n}{plt}\PY{o}{.}\PY{n}{axis}\PY{p}{(}\PY{l+s+s1}{\PYZsq{}}\PY{l+s+s1}{off}\PY{l+s+s1}{\PYZsq{}}\PY{p}{)}

\PY{n}{plt}\PY{o}{.}\PY{n}{show}\PY{p}{(}\PY{p}{)}
\end{Verbatim}
\end{tcolorbox}

    \begin{center}
    \adjustimage{max size={0.9\linewidth}{0.9\paperheight}}{output_16_0.png}
    \end{center}
    { \hspace*{\fill} \\}
    
    Next, complete the function \textbf{\texttt{conv\_fast}} in
\textbf{\texttt{filters.py}} using \texttt{zero\_pad}. Run the code
below to compare the outputs by the two implementations.
\texttt{conv\_fast} should run noticeably faster than
\texttt{conv\_nested}.

    \begin{tcolorbox}[breakable, size=fbox, boxrule=1pt, pad at break*=1mm,colback=cellbackground, colframe=cellborder]
\prompt{In}{incolor}{171}{\boxspacing}
\begin{Verbatim}[commandchars=\\\{\}]
\PY{k+kn}{from} \PY{n+nn}{filters} \PY{k+kn}{import} \PY{n}{conv\PYZus{}fast}

\PY{n}{t0} \PY{o}{=} \PY{n}{time}\PY{p}{(}\PY{p}{)}
\PY{n}{out\PYZus{}fast} \PY{o}{=} \PY{n}{conv\PYZus{}fast}\PY{p}{(}\PY{n}{img}\PY{p}{,} \PY{n}{kernel}\PY{p}{)}
\PY{n}{t1} \PY{o}{=} \PY{n}{time}\PY{p}{(}\PY{p}{)}
\PY{n}{out\PYZus{}nested} \PY{o}{=} \PY{n}{conv\PYZus{}nested}\PY{p}{(}\PY{n}{img}\PY{p}{,} \PY{n}{kernel}\PY{p}{)}
\PY{n}{t2} \PY{o}{=} \PY{n}{time}\PY{p}{(}\PY{p}{)}

\PY{c+c1}{\PYZsh{} Compare the running time of the two implementations}
\PY{n+nb}{print}\PY{p}{(}\PY{l+s+s2}{\PYZdq{}}\PY{l+s+s2}{conv\PYZus{}nested: took }\PY{l+s+si}{\PYZpc{}f}\PY{l+s+s2}{ seconds.}\PY{l+s+s2}{\PYZdq{}} \PY{o}{\PYZpc{}} \PY{p}{(}\PY{n}{t2} \PY{o}{\PYZhy{}} \PY{n}{t1}\PY{p}{)}\PY{p}{)}
\PY{n+nb}{print}\PY{p}{(}\PY{l+s+s2}{\PYZdq{}}\PY{l+s+s2}{conv\PYZus{}fast: took }\PY{l+s+si}{\PYZpc{}f}\PY{l+s+s2}{ seconds.}\PY{l+s+s2}{\PYZdq{}} \PY{o}{\PYZpc{}} \PY{p}{(}\PY{n}{t1} \PY{o}{\PYZhy{}} \PY{n}{t0}\PY{p}{)}\PY{p}{)}

\PY{c+c1}{\PYZsh{} Plot conv\PYZus{}nested output}
\PY{n}{plt}\PY{o}{.}\PY{n}{subplot}\PY{p}{(}\PY{l+m+mi}{1}\PY{p}{,}\PY{l+m+mi}{2}\PY{p}{,}\PY{l+m+mi}{1}\PY{p}{)}
\PY{n}{plt}\PY{o}{.}\PY{n}{imshow}\PY{p}{(}\PY{n}{out\PYZus{}nested}\PY{p}{)}
\PY{n}{plt}\PY{o}{.}\PY{n}{title}\PY{p}{(}\PY{l+s+s1}{\PYZsq{}}\PY{l+s+s1}{conv\PYZus{}nested}\PY{l+s+s1}{\PYZsq{}}\PY{p}{)}
\PY{n}{plt}\PY{o}{.}\PY{n}{axis}\PY{p}{(}\PY{l+s+s1}{\PYZsq{}}\PY{l+s+s1}{off}\PY{l+s+s1}{\PYZsq{}}\PY{p}{)}

\PY{c+c1}{\PYZsh{} Plot conv\PYZus{}fast output}
\PY{n}{plt}\PY{o}{.}\PY{n}{subplot}\PY{p}{(}\PY{l+m+mi}{1}\PY{p}{,}\PY{l+m+mi}{2}\PY{p}{,}\PY{l+m+mi}{2}\PY{p}{)}
\PY{n}{plt}\PY{o}{.}\PY{n}{imshow}\PY{p}{(}\PY{n}{out\PYZus{}fast}\PY{p}{)}
\PY{n}{plt}\PY{o}{.}\PY{n}{title}\PY{p}{(}\PY{l+s+s1}{\PYZsq{}}\PY{l+s+s1}{conv\PYZus{}fast}\PY{l+s+s1}{\PYZsq{}}\PY{p}{)}
\PY{n}{plt}\PY{o}{.}\PY{n}{axis}\PY{p}{(}\PY{l+s+s1}{\PYZsq{}}\PY{l+s+s1}{off}\PY{l+s+s1}{\PYZsq{}}\PY{p}{)}

\PY{c+c1}{\PYZsh{} Make sure that the two outputs are the same}
\PY{k}{if} \PY{o+ow}{not} \PY{p}{(}\PY{n}{np}\PY{o}{.}\PY{n}{max}\PY{p}{(}\PY{n}{out\PYZus{}fast} \PY{o}{\PYZhy{}} \PY{n}{out\PYZus{}nested}\PY{p}{)} \PY{o}{\PYZlt{}} \PY{l+m+mf}{1e\PYZhy{}10}\PY{p}{)}\PY{p}{:}
    \PY{n+nb}{print}\PY{p}{(}\PY{l+s+s2}{\PYZdq{}}\PY{l+s+s2}{Different outputs! Check your implementation.}\PY{l+s+s2}{\PYZdq{}}\PY{p}{)}
\end{Verbatim}
\end{tcolorbox}

    \begin{Verbatim}[commandchars=\\\{\}]
conv\_nested: took 1.616058 seconds.
conv\_fast: took 1.073712 seconds.
    \end{Verbatim}

    \begin{center}
    \adjustimage{max size={0.9\linewidth}{0.9\paperheight}}{output_18_1.png}
    \end{center}
    { \hspace*{\fill} \\}
    
    \begin{center}\rule{0.5\linewidth}{0.5pt}\end{center}

\hypertarget{part-2-cross-correlation}{%
\subsection{Part 2: Cross-correlation}\label{part-2-cross-correlation}}

Cross-correlation of two 2D signals \(f\) and \(g\) is defined as
follows:
\[(f\star{g})[m,n]=\sum_{i=-\infty}^\infty\sum_{j=-\infty}^\infty f[i,j]\cdot g[m + i,n + j]\]

    \hypertarget{template-matching-with-cross-correlation-12-points}{%
\subsubsection{2.1 Template Matching with Cross-correlation (12
points)}\label{template-matching-with-cross-correlation-12-points}}

Suppose that you are a clerk at a grocery store. One of your
responsibilites is to check the shelves periodically and stock them up
whenever there are sold-out items. You got tired of this laborious task
and decided to build a computer vision system that keeps track of the
items on the shelf.

Luckily, you have learned in CS131 that cross-correlation can be used
for template matching: a template \(g\) is multiplied with regions of a
larger image \(f\) to measure how similar each region is to the
template.

The template of a product (\texttt{template.jpg}) and the image of shelf
(\texttt{shelf.jpg}) is provided. We will use cross-correlation to find
the product in the shelf.

Implement \textbf{\texttt{cross\_correlation}} function in
\textbf{\texttt{filters.py}} and run the code below.

\emph{- Hint: you may use the \texttt{conv\_fast} function you
implemented in the previous question.}

    \begin{tcolorbox}[breakable, size=fbox, boxrule=1pt, pad at break*=1mm,colback=cellbackground, colframe=cellborder]
\prompt{In}{incolor}{172}{\boxspacing}
\begin{Verbatim}[commandchars=\\\{\}]
\PY{k+kn}{from} \PY{n+nn}{filters} \PY{k+kn}{import} \PY{n}{cross\PYZus{}correlation}

\PY{c+c1}{\PYZsh{} Load template and image in grayscale}
\PY{n}{img} \PY{o}{=} \PY{n}{io}\PY{o}{.}\PY{n}{imread}\PY{p}{(}\PY{l+s+s1}{\PYZsq{}}\PY{l+s+s1}{shelf.jpg}\PY{l+s+s1}{\PYZsq{}}\PY{p}{)}
\PY{n}{img\PYZus{}gray} \PY{o}{=} \PY{n}{io}\PY{o}{.}\PY{n}{imread}\PY{p}{(}\PY{l+s+s1}{\PYZsq{}}\PY{l+s+s1}{shelf.jpg}\PY{l+s+s1}{\PYZsq{}}\PY{p}{,} \PY{n}{as\PYZus{}gray}\PY{o}{=}\PY{k+kc}{True}\PY{p}{)}
\PY{n}{temp} \PY{o}{=} \PY{n}{io}\PY{o}{.}\PY{n}{imread}\PY{p}{(}\PY{l+s+s1}{\PYZsq{}}\PY{l+s+s1}{template.jpg}\PY{l+s+s1}{\PYZsq{}}\PY{p}{)}
\PY{n}{temp\PYZus{}gray} \PY{o}{=} \PY{n}{io}\PY{o}{.}\PY{n}{imread}\PY{p}{(}\PY{l+s+s1}{\PYZsq{}}\PY{l+s+s1}{template.jpg}\PY{l+s+s1}{\PYZsq{}}\PY{p}{,} \PY{n}{as\PYZus{}gray}\PY{o}{=}\PY{k+kc}{True}\PY{p}{)}

\PY{c+c1}{\PYZsh{} Perform cross\PYZhy{}correlation between the image and the template}
\PY{n}{out} \PY{o}{=} \PY{n}{cross\PYZus{}correlation}\PY{p}{(}\PY{n}{img\PYZus{}gray}\PY{p}{,} \PY{n}{temp\PYZus{}gray}\PY{p}{)}

\PY{c+c1}{\PYZsh{} Find the location with maximum similarity}
\PY{n}{y}\PY{p}{,}\PY{n}{x} \PY{o}{=} \PY{p}{(}\PY{n}{np}\PY{o}{.}\PY{n}{unravel\PYZus{}index}\PY{p}{(}\PY{n}{out}\PY{o}{.}\PY{n}{argmax}\PY{p}{(}\PY{p}{)}\PY{p}{,} \PY{n}{out}\PY{o}{.}\PY{n}{shape}\PY{p}{)}\PY{p}{)}

\PY{c+c1}{\PYZsh{} Display product template}
\PY{n}{plt}\PY{o}{.}\PY{n}{figure}\PY{p}{(}\PY{n}{figsize}\PY{o}{=}\PY{p}{(}\PY{l+m+mi}{25}\PY{p}{,}\PY{l+m+mi}{20}\PY{p}{)}\PY{p}{)}
\PY{n}{plt}\PY{o}{.}\PY{n}{subplot}\PY{p}{(}\PY{l+m+mi}{3}\PY{p}{,} \PY{l+m+mi}{1}\PY{p}{,} \PY{l+m+mi}{1}\PY{p}{)}
\PY{n}{plt}\PY{o}{.}\PY{n}{imshow}\PY{p}{(}\PY{n}{temp}\PY{p}{)}
\PY{n}{plt}\PY{o}{.}\PY{n}{title}\PY{p}{(}\PY{l+s+s1}{\PYZsq{}}\PY{l+s+s1}{Template}\PY{l+s+s1}{\PYZsq{}}\PY{p}{)}
\PY{n}{plt}\PY{o}{.}\PY{n}{axis}\PY{p}{(}\PY{l+s+s1}{\PYZsq{}}\PY{l+s+s1}{off}\PY{l+s+s1}{\PYZsq{}}\PY{p}{)}

\PY{c+c1}{\PYZsh{} Display cross\PYZhy{}correlation output}
\PY{n}{plt}\PY{o}{.}\PY{n}{subplot}\PY{p}{(}\PY{l+m+mi}{3}\PY{p}{,} \PY{l+m+mi}{1}\PY{p}{,} \PY{l+m+mi}{2}\PY{p}{)}
\PY{n}{plt}\PY{o}{.}\PY{n}{imshow}\PY{p}{(}\PY{n}{out}\PY{p}{)}
\PY{n}{plt}\PY{o}{.}\PY{n}{title}\PY{p}{(}\PY{l+s+s1}{\PYZsq{}}\PY{l+s+s1}{Cross\PYZhy{}correlation (white means more correlated)}\PY{l+s+s1}{\PYZsq{}}\PY{p}{)}
\PY{n}{plt}\PY{o}{.}\PY{n}{axis}\PY{p}{(}\PY{l+s+s1}{\PYZsq{}}\PY{l+s+s1}{off}\PY{l+s+s1}{\PYZsq{}}\PY{p}{)}

\PY{c+c1}{\PYZsh{} Display image}
\PY{n}{plt}\PY{o}{.}\PY{n}{subplot}\PY{p}{(}\PY{l+m+mi}{3}\PY{p}{,} \PY{l+m+mi}{1}\PY{p}{,} \PY{l+m+mi}{3}\PY{p}{)}
\PY{n}{plt}\PY{o}{.}\PY{n}{imshow}\PY{p}{(}\PY{n}{img}\PY{p}{)}
\PY{n}{plt}\PY{o}{.}\PY{n}{title}\PY{p}{(}\PY{l+s+s1}{\PYZsq{}}\PY{l+s+s1}{Result (blue marker on the detected location)}\PY{l+s+s1}{\PYZsq{}}\PY{p}{)}
\PY{n}{plt}\PY{o}{.}\PY{n}{axis}\PY{p}{(}\PY{l+s+s1}{\PYZsq{}}\PY{l+s+s1}{off}\PY{l+s+s1}{\PYZsq{}}\PY{p}{)}

\PY{c+c1}{\PYZsh{} Draw marker at detected location}
\PY{n}{plt}\PY{o}{.}\PY{n}{plot}\PY{p}{(}\PY{n}{x}\PY{p}{,} \PY{n}{y}\PY{p}{,} \PY{l+s+s1}{\PYZsq{}}\PY{l+s+s1}{bx}\PY{l+s+s1}{\PYZsq{}}\PY{p}{,} \PY{n}{ms}\PY{o}{=}\PY{l+m+mi}{40}\PY{p}{,} \PY{n}{mew}\PY{o}{=}\PY{l+m+mi}{10}\PY{p}{)}
\PY{n}{plt}\PY{o}{.}\PY{n}{show}\PY{p}{(}\PY{p}{)}
\end{Verbatim}
\end{tcolorbox}

    \begin{center}
    \adjustimage{max size={0.9\linewidth}{0.9\paperheight}}{output_21_0.png}
    \end{center}
    { \hspace*{\fill} \\}
    
    \hypertarget{interpretation}{%
\paragraph{Interpretation}\label{interpretation}}

How does the output of cross-correlation filter look? Explain what
problems there might be with using a raw template as a filter.

    \textbf{Your Answer:} The output of the cross-correlation filter is very
spotty and ambiguous in determining the brightest spot representing the
template cereal box. Using a raw template as a filter can create issues
because correlations with brighter images are always larger, regardless
of the chosen kernel (bright white spots have higher correlation). This
creates many false positives in bright sections of the image despite the
target kernel being absent. For this specific image, it seems as if the
white space between the rows of cereal boxes produces error. In
addition, cross-correlation is very sensitive to noise and this
contributes to the incorret comparison of images. The following
solutions of zero-mean cross-correlation and normalized
cross-correlation make the prediction more accurate because they more
precisely evaluate the degree of similarity between two compared images.
Rather than looking at the raw pixels, they normalize each pixel to
account for true similarity rather than highest brightness.

    \begin{center}\rule{0.5\linewidth}{0.5pt}\end{center}

\hypertarget{zero-mean-cross-correlation-6-points}{%
\subsubsection{2.2 Zero-mean cross-correlation (6
points)}\label{zero-mean-cross-correlation-6-points}}

A solution to this problem is to subtract the mean value of the template
so that it has zero mean.

Implement \textbf{\texttt{zero\_mean\_cross\_correlation}} function in
\textbf{\texttt{filters.py}} and run the code below.

\textbf{If your implementation is correct, you should see the blue cross
centered over the correct cereal box.}

    \begin{tcolorbox}[breakable, size=fbox, boxrule=1pt, pad at break*=1mm,colback=cellbackground, colframe=cellborder]
\prompt{In}{incolor}{173}{\boxspacing}
\begin{Verbatim}[commandchars=\\\{\}]
\PY{k+kn}{from} \PY{n+nn}{filters} \PY{k+kn}{import} \PY{n}{zero\PYZus{}mean\PYZus{}cross\PYZus{}correlation}

\PY{c+c1}{\PYZsh{} Perform cross\PYZhy{}correlation between the image and the template}
\PY{n}{out} \PY{o}{=} \PY{n}{zero\PYZus{}mean\PYZus{}cross\PYZus{}correlation}\PY{p}{(}\PY{n}{img\PYZus{}gray}\PY{p}{,} \PY{n}{temp\PYZus{}gray}\PY{p}{)}

\PY{c+c1}{\PYZsh{} Find the location with maximum similarity}
\PY{n}{y}\PY{p}{,}\PY{n}{x} \PY{o}{=} \PY{n}{np}\PY{o}{.}\PY{n}{unravel\PYZus{}index}\PY{p}{(}\PY{n}{out}\PY{o}{.}\PY{n}{argmax}\PY{p}{(}\PY{p}{)}\PY{p}{,} \PY{n}{out}\PY{o}{.}\PY{n}{shape}\PY{p}{)}

\PY{c+c1}{\PYZsh{} Display product template}
\PY{n}{plt}\PY{o}{.}\PY{n}{figure}\PY{p}{(}\PY{n}{figsize}\PY{o}{=}\PY{p}{(}\PY{l+m+mi}{30}\PY{p}{,}\PY{l+m+mi}{20}\PY{p}{)}\PY{p}{)}
\PY{n}{plt}\PY{o}{.}\PY{n}{subplot}\PY{p}{(}\PY{l+m+mi}{3}\PY{p}{,} \PY{l+m+mi}{1}\PY{p}{,} \PY{l+m+mi}{1}\PY{p}{)}
\PY{n}{plt}\PY{o}{.}\PY{n}{imshow}\PY{p}{(}\PY{n}{temp}\PY{p}{)}
\PY{n}{plt}\PY{o}{.}\PY{n}{title}\PY{p}{(}\PY{l+s+s1}{\PYZsq{}}\PY{l+s+s1}{Template}\PY{l+s+s1}{\PYZsq{}}\PY{p}{)}
\PY{n}{plt}\PY{o}{.}\PY{n}{axis}\PY{p}{(}\PY{l+s+s1}{\PYZsq{}}\PY{l+s+s1}{off}\PY{l+s+s1}{\PYZsq{}}\PY{p}{)}

\PY{c+c1}{\PYZsh{} Display cross\PYZhy{}correlation output}
\PY{n}{plt}\PY{o}{.}\PY{n}{subplot}\PY{p}{(}\PY{l+m+mi}{3}\PY{p}{,} \PY{l+m+mi}{1}\PY{p}{,} \PY{l+m+mi}{2}\PY{p}{)}
\PY{n}{plt}\PY{o}{.}\PY{n}{imshow}\PY{p}{(}\PY{n}{out}\PY{p}{)}
\PY{n}{plt}\PY{o}{.}\PY{n}{title}\PY{p}{(}\PY{l+s+s1}{\PYZsq{}}\PY{l+s+s1}{Cross\PYZhy{}correlation (white means more correlated)}\PY{l+s+s1}{\PYZsq{}}\PY{p}{)}
\PY{n}{plt}\PY{o}{.}\PY{n}{axis}\PY{p}{(}\PY{l+s+s1}{\PYZsq{}}\PY{l+s+s1}{off}\PY{l+s+s1}{\PYZsq{}}\PY{p}{)}

\PY{c+c1}{\PYZsh{} Display image}
\PY{n}{plt}\PY{o}{.}\PY{n}{subplot}\PY{p}{(}\PY{l+m+mi}{3}\PY{p}{,} \PY{l+m+mi}{1}\PY{p}{,} \PY{l+m+mi}{3}\PY{p}{)}
\PY{n}{plt}\PY{o}{.}\PY{n}{imshow}\PY{p}{(}\PY{n}{img}\PY{p}{)}
\PY{n}{plt}\PY{o}{.}\PY{n}{title}\PY{p}{(}\PY{l+s+s1}{\PYZsq{}}\PY{l+s+s1}{Result (blue marker on the detected location)}\PY{l+s+s1}{\PYZsq{}}\PY{p}{)}
\PY{n}{plt}\PY{o}{.}\PY{n}{axis}\PY{p}{(}\PY{l+s+s1}{\PYZsq{}}\PY{l+s+s1}{off}\PY{l+s+s1}{\PYZsq{}}\PY{p}{)}

\PY{c+c1}{\PYZsh{} Draw marker at detected location}
\PY{n}{plt}\PY{o}{.}\PY{n}{plot}\PY{p}{(}\PY{n}{x}\PY{p}{,} \PY{n}{y}\PY{p}{,} \PY{l+s+s1}{\PYZsq{}}\PY{l+s+s1}{bx}\PY{l+s+s1}{\PYZsq{}}\PY{p}{,} \PY{n}{ms}\PY{o}{=}\PY{l+m+mi}{40}\PY{p}{,} \PY{n}{mew}\PY{o}{=}\PY{l+m+mi}{10}\PY{p}{)}
\PY{n}{plt}\PY{o}{.}\PY{n}{show}\PY{p}{(}\PY{p}{)}
\end{Verbatim}
\end{tcolorbox}

    \begin{center}
    \adjustimage{max size={0.9\linewidth}{0.9\paperheight}}{output_25_0.png}
    \end{center}
    { \hspace*{\fill} \\}
    
    You can also determine whether the product is present with appropriate
scaling and thresholding.

    \begin{tcolorbox}[breakable, size=fbox, boxrule=1pt, pad at break*=1mm,colback=cellbackground, colframe=cellborder]
\prompt{In}{incolor}{174}{\boxspacing}
\begin{Verbatim}[commandchars=\\\{\}]
\PY{k}{def} \PY{n+nf}{check\PYZus{}product\PYZus{}on\PYZus{}shelf}\PY{p}{(}\PY{n}{shelf}\PY{p}{,} \PY{n}{product}\PY{p}{)}\PY{p}{:}
    \PY{n}{out} \PY{o}{=} \PY{n}{zero\PYZus{}mean\PYZus{}cross\PYZus{}correlation}\PY{p}{(}\PY{n}{shelf}\PY{p}{,} \PY{n}{product}\PY{p}{)}
    
    \PY{c+c1}{\PYZsh{} Scale output by the size of the template}
    \PY{n}{out} \PY{o}{=} \PY{n}{out} \PY{o}{/} \PY{n+nb}{float}\PY{p}{(}\PY{n}{product}\PY{o}{.}\PY{n}{shape}\PY{p}{[}\PY{l+m+mi}{0}\PY{p}{]}\PY{o}{*}\PY{n}{product}\PY{o}{.}\PY{n}{shape}\PY{p}{[}\PY{l+m+mi}{1}\PY{p}{]}\PY{p}{)}
    
    \PY{c+c1}{\PYZsh{} Threshold output (this is arbitrary, you would need to tune the threshold for a real application)}
    \PY{n}{out} \PY{o}{=} \PY{n}{out} \PY{o}{\PYZgt{}} \PY{l+m+mf}{0.025}
    
    \PY{k}{if} \PY{n}{np}\PY{o}{.}\PY{n}{sum}\PY{p}{(}\PY{n}{out}\PY{p}{)} \PY{o}{\PYZgt{}} \PY{l+m+mi}{0}\PY{p}{:}
        \PY{n+nb}{print}\PY{p}{(}\PY{l+s+s1}{\PYZsq{}}\PY{l+s+s1}{The product is on the shelf}\PY{l+s+s1}{\PYZsq{}}\PY{p}{)}
    \PY{k}{else}\PY{p}{:}
        \PY{n+nb}{print}\PY{p}{(}\PY{l+s+s1}{\PYZsq{}}\PY{l+s+s1}{The product is not on the shelf}\PY{l+s+s1}{\PYZsq{}}\PY{p}{)}

\PY{c+c1}{\PYZsh{} Load image of the shelf without the product}
\PY{n}{img2} \PY{o}{=} \PY{n}{io}\PY{o}{.}\PY{n}{imread}\PY{p}{(}\PY{l+s+s1}{\PYZsq{}}\PY{l+s+s1}{shelf\PYZus{}soldout.jpg}\PY{l+s+s1}{\PYZsq{}}\PY{p}{)}
\PY{n}{img2\PYZus{}gray} \PY{o}{=} \PY{n}{io}\PY{o}{.}\PY{n}{imread}\PY{p}{(}\PY{l+s+s1}{\PYZsq{}}\PY{l+s+s1}{shelf\PYZus{}soldout.jpg}\PY{l+s+s1}{\PYZsq{}}\PY{p}{,} \PY{n}{as\PYZus{}gray}\PY{o}{=}\PY{k+kc}{True}\PY{p}{)}

\PY{n}{plt}\PY{o}{.}\PY{n}{imshow}\PY{p}{(}\PY{n}{img}\PY{p}{)}
\PY{n}{plt}\PY{o}{.}\PY{n}{axis}\PY{p}{(}\PY{l+s+s1}{\PYZsq{}}\PY{l+s+s1}{off}\PY{l+s+s1}{\PYZsq{}}\PY{p}{)}
\PY{n}{plt}\PY{o}{.}\PY{n}{show}\PY{p}{(}\PY{p}{)}
\PY{n}{check\PYZus{}product\PYZus{}on\PYZus{}shelf}\PY{p}{(}\PY{n}{img\PYZus{}gray}\PY{p}{,} \PY{n}{temp\PYZus{}gray}\PY{p}{)}

\PY{n}{plt}\PY{o}{.}\PY{n}{imshow}\PY{p}{(}\PY{n}{img2}\PY{p}{)}
\PY{n}{plt}\PY{o}{.}\PY{n}{axis}\PY{p}{(}\PY{l+s+s1}{\PYZsq{}}\PY{l+s+s1}{off}\PY{l+s+s1}{\PYZsq{}}\PY{p}{)}
\PY{n}{plt}\PY{o}{.}\PY{n}{show}\PY{p}{(}\PY{p}{)}
\PY{n}{check\PYZus{}product\PYZus{}on\PYZus{}shelf}\PY{p}{(}\PY{n}{img2\PYZus{}gray}\PY{p}{,} \PY{n}{temp\PYZus{}gray}\PY{p}{)}
\end{Verbatim}
\end{tcolorbox}

    \begin{center}
    \adjustimage{max size={0.9\linewidth}{0.9\paperheight}}{output_27_0.png}
    \end{center}
    { \hspace*{\fill} \\}
    
    \begin{Verbatim}[commandchars=\\\{\}]
The product is on the shelf
    \end{Verbatim}

    \begin{center}
    \adjustimage{max size={0.9\linewidth}{0.9\paperheight}}{output_27_2.png}
    \end{center}
    { \hspace*{\fill} \\}
    
    \begin{Verbatim}[commandchars=\\\{\}]
The product is not on the shelf
    \end{Verbatim}

    \begin{center}\rule{0.5\linewidth}{0.5pt}\end{center}

\hypertarget{normalized-cross-correlation-12-points}{%
\subsubsection{2.3 Normalized Cross-correlation (12
points)}\label{normalized-cross-correlation-12-points}}

One day the light near the shelf goes out and the product tracker starts
to malfunction. The \texttt{zero\_mean\_cross\_correlation} is not
robust to change in lighting condition. The code below demonstrates
this.

    \begin{tcolorbox}[breakable, size=fbox, boxrule=1pt, pad at break*=1mm,colback=cellbackground, colframe=cellborder]
\prompt{In}{incolor}{175}{\boxspacing}
\begin{Verbatim}[commandchars=\\\{\}]
\PY{k+kn}{from} \PY{n+nn}{filters} \PY{k+kn}{import} \PY{n}{normalized\PYZus{}cross\PYZus{}correlation}

\PY{c+c1}{\PYZsh{} Load image}
\PY{n}{img} \PY{o}{=} \PY{n}{io}\PY{o}{.}\PY{n}{imread}\PY{p}{(}\PY{l+s+s1}{\PYZsq{}}\PY{l+s+s1}{shelf\PYZus{}dark.jpg}\PY{l+s+s1}{\PYZsq{}}\PY{p}{)}
\PY{n}{img\PYZus{}gray} \PY{o}{=} \PY{n}{io}\PY{o}{.}\PY{n}{imread}\PY{p}{(}\PY{l+s+s1}{\PYZsq{}}\PY{l+s+s1}{shelf\PYZus{}dark.jpg}\PY{l+s+s1}{\PYZsq{}}\PY{p}{,} \PY{n}{as\PYZus{}gray}\PY{o}{=}\PY{k+kc}{True}\PY{p}{)}

\PY{c+c1}{\PYZsh{} Perform cross\PYZhy{}correlation between the image and the template}
\PY{n}{out} \PY{o}{=} \PY{n}{zero\PYZus{}mean\PYZus{}cross\PYZus{}correlation}\PY{p}{(}\PY{n}{img\PYZus{}gray}\PY{p}{,} \PY{n}{temp\PYZus{}gray}\PY{p}{)}

\PY{c+c1}{\PYZsh{} Find the location with maximum similarity}
\PY{n}{y}\PY{p}{,} \PY{n}{x} \PY{o}{=} \PY{n}{np}\PY{o}{.}\PY{n}{unravel\PYZus{}index}\PY{p}{(}\PY{n}{out}\PY{o}{.}\PY{n}{argmax}\PY{p}{(}\PY{p}{)}\PY{p}{,} \PY{n}{out}\PY{o}{.}\PY{n}{shape}\PY{p}{)}

\PY{c+c1}{\PYZsh{} Display image}
\PY{n}{plt}\PY{o}{.}\PY{n}{imshow}\PY{p}{(}\PY{n}{img}\PY{p}{)}
\PY{n}{plt}\PY{o}{.}\PY{n}{title}\PY{p}{(}\PY{l+s+s1}{\PYZsq{}}\PY{l+s+s1}{Result (red marker on the detected location)}\PY{l+s+s1}{\PYZsq{}}\PY{p}{)}
\PY{n}{plt}\PY{o}{.}\PY{n}{axis}\PY{p}{(}\PY{l+s+s1}{\PYZsq{}}\PY{l+s+s1}{off}\PY{l+s+s1}{\PYZsq{}}\PY{p}{)}

\PY{c+c1}{\PYZsh{} Draw marker at detcted location}
\PY{n}{plt}\PY{o}{.}\PY{n}{plot}\PY{p}{(}\PY{n}{x}\PY{p}{,} \PY{n}{y}\PY{p}{,} \PY{l+s+s1}{\PYZsq{}}\PY{l+s+s1}{rx}\PY{l+s+s1}{\PYZsq{}}\PY{p}{,} \PY{n}{ms}\PY{o}{=}\PY{l+m+mi}{25}\PY{p}{,} \PY{n}{mew}\PY{o}{=}\PY{l+m+mi}{5}\PY{p}{)}
\PY{n}{plt}\PY{o}{.}\PY{n}{show}\PY{p}{(}\PY{p}{)}
\end{Verbatim}
\end{tcolorbox}

    \begin{center}
    \adjustimage{max size={0.9\linewidth}{0.9\paperheight}}{output_29_0.png}
    \end{center}
    { \hspace*{\fill} \\}
    
    A solution is to normalize the pixels of the image and template at every
step before comparing them. This is called \textbf{normalized
cross-correlation}.

The mathematical definition for normalized cross-correlation of \(f\)
and template \(g\) is:
\[(f\star{g})[m,n]=\sum_{i,j} \frac{f[i,j]-\overline{f_{m,n}}}{\sigma_{f_{m,n}}} \cdot \frac{g[i-m,j-n]-\overline{g}}{\sigma_g}\]

where: - \(f_{m,n}\) is the patch image at position \((m,n)\) -
\(\overline{f_{m,n}}\) is the mean of the patch image \(f_{m,n}\) -
\(\sigma_{f_{m,n}}\) is the standard deviation of the patch image
\(f_{m,n}\) - \(\overline{g}\) is the mean of the template \(g\) -
\(\sigma_g\) is the standard deviation of the template \(g\)

Implement \textbf{\texttt{normalized\_cross\_correlation}} function in
\textbf{\texttt{filters.py}} and run the code below.

    \begin{tcolorbox}[breakable, size=fbox, boxrule=1pt, pad at break*=1mm,colback=cellbackground, colframe=cellborder]
\prompt{In}{incolor}{176}{\boxspacing}
\begin{Verbatim}[commandchars=\\\{\}]
\PY{k+kn}{from} \PY{n+nn}{filters} \PY{k+kn}{import} \PY{n}{normalized\PYZus{}cross\PYZus{}correlation}

\PY{c+c1}{\PYZsh{} Perform normalized cross\PYZhy{}correlation between the image and the template}
\PY{n}{out} \PY{o}{=} \PY{n}{normalized\PYZus{}cross\PYZus{}correlation}\PY{p}{(}\PY{n}{img\PYZus{}gray}\PY{p}{,} \PY{n}{temp\PYZus{}gray}\PY{p}{)}
\PY{c+c1}{\PYZsh{}plt.imshow(img\PYZus{}gray)}
\PY{c+c1}{\PYZsh{}plt.imshow(temp\PYZus{}gray)}

\PY{c+c1}{\PYZsh{} Find the location with maximum similarity}
\PY{n}{y}\PY{p}{,} \PY{n}{x} \PY{o}{=} \PY{n}{np}\PY{o}{.}\PY{n}{unravel\PYZus{}index}\PY{p}{(}\PY{n}{out}\PY{o}{.}\PY{n}{argmax}\PY{p}{(}\PY{p}{)}\PY{p}{,} \PY{n}{out}\PY{o}{.}\PY{n}{shape}\PY{p}{)}

\PY{c+c1}{\PYZsh{} Display image}
\PY{n}{plt}\PY{o}{.}\PY{n}{imshow}\PY{p}{(}\PY{n}{img\PYZus{}gray}\PY{p}{)}
\PY{n}{plt}\PY{o}{.}\PY{n}{title}\PY{p}{(}\PY{l+s+s1}{\PYZsq{}}\PY{l+s+s1}{Result (red marker on the detected location)}\PY{l+s+s1}{\PYZsq{}}\PY{p}{)}
\PY{n}{plt}\PY{o}{.}\PY{n}{axis}\PY{p}{(}\PY{l+s+s1}{\PYZsq{}}\PY{l+s+s1}{off}\PY{l+s+s1}{\PYZsq{}}\PY{p}{)}

\PY{c+c1}{\PYZsh{} Draw marker at detcted location}
\PY{n}{plt}\PY{o}{.}\PY{n}{plot}\PY{p}{(}\PY{n}{x}\PY{p}{,} \PY{n}{y}\PY{p}{,} \PY{l+s+s1}{\PYZsq{}}\PY{l+s+s1}{rx}\PY{l+s+s1}{\PYZsq{}}\PY{p}{,} \PY{n}{ms}\PY{o}{=}\PY{l+m+mi}{25}\PY{p}{,} \PY{n}{mew}\PY{o}{=}\PY{l+m+mi}{5}\PY{p}{)}
\PY{n}{plt}\PY{o}{.}\PY{n}{show}\PY{p}{(}\PY{p}{)}
\end{Verbatim}
\end{tcolorbox}

    \begin{center}
    \adjustimage{max size={0.9\linewidth}{0.9\paperheight}}{output_31_0.png}
    \end{center}
    { \hspace*{\fill} \\}
    
    \hypertarget{part-3-separable-filters}{%
\subsection{Part 3: Separable Filters}\label{part-3-separable-filters}}

    \hypertarget{theory-10-points}{%
\subsubsection{3.1 Theory (10 points)}\label{theory-10-points}}

Consider an \(M_1\times{N_1}\) image \(I\) and an \(M_2\times{N_2}\)
filter \(F\). A filter \(F\) is \textbf{separable} if it can be written
as a product of two 1D filters: \(F=F_1F_2\).

For example, \[F=
\begin{bmatrix}
1 & -1 \\
1 & -1
\end{bmatrix}
\] can be written as a matrix product of \[F_1=
\begin{bmatrix}
1  \\
1
\end{bmatrix},
F_2=
\begin{bmatrix}
1 & -1
\end{bmatrix}
\] Therefore \(F\) is a separable filter.

Prove that for any separable filter \(F=F_1F_2\), \[I*F=(I*F_1)*F_2\]
where \(*\) is the convolution operation.

    \textbf{Answer:}

By 1D convolution,
\(I * F_1 = F_1 * I = \sum_{i=-\infty}^{\infty}\sum_{j=-\infty}^{\infty}F_1[i] \cdot I[m - i, n - j]\)

\((I * F)[m, n] = (F * I)[m, n] = \sum_{i=-\infty}^{\infty}\sum_{j=-\infty}^{\infty}F[i, j] \cdot I[m - i, n - j]\)

\$ =
\sum\emph{\{i=-\infty\}\^{}\{\infty\}\sum}\{j=-\infty\}\^{}\{\infty\}(F\_1
F\_2){[}i, j{]} \cdot I{[}m - i, n - j{]}\$

\$ =
\sum\emph{\{i=-\infty\}\^{}\{\infty\}\sum}\{j=-\infty\}\^{}\{\infty\}F\_1{[}i,
0{]} \cdot F\_2{[}0, j{]} \cdot I{[}m - i, n - j{]}\$

\$ = \sum\_\{j=-\infty\}\^{}\{\infty\}F\_2{[}0, j{]}\$
\((\sum_{i=-\infty}^{\infty}I[m - i, n - j] \cdot F_1[i, 0])\)

\((I * F_1)[m, n] = \sum_{i=-\infty}^{\infty}\sum_{j=-\infty}^{\infty}F_1[i, j] \cdot I[m - i, n - j]\)

\$ = \sum\_\{i=-\infty\}\^{}\{\infty\}F\_1{[}i, 0{]} \cdot I{[}m - i,
n{]}\$

\$ I\_2{[}m, n{]} = (I * F\_1){[}m, n{]} \$

\((I_2 * F_2)[m, n] = \sum_{i=-\infty}^{\infty}\sum_{j=-\infty}^{\infty}F[i, j] \cdot I_2[m - i, n - j]\)

\$ = \sum\_\{j=-\infty\}\^{}\{\infty\}F\_2{[}0, j{]}\$
\((\sum_{i=-\infty}^{\infty}I[m - i, n - j] \cdot F_1[i, 0])\)

    \hypertarget{complexity-comparison-10-points}{%
\subsubsection{3.2 Complexity comparison (10
points)}\label{complexity-comparison-10-points}}

Consider an \(M_1\times{N_1}\) image \(I\) and an \(M_2\times{N_2}\)
filter \(F\) that is separable (i.e.~\(F=F_1F_2\)).

\begin{enumerate}
\def\labelenumi{(\roman{enumi})}
\tightlist
\item
  How many multiplication operations do you need to do a direct 2D
  convolution (i.e.~\(I*F\))?
\item
  How many multiplication operations do you need to do 1D convolutions
  on rows and columns (i.e.~\((I*F_1)*F_2\))?
\item
  Use Big-O notation, written with respet to the dimensions \(M_1\),
  \(N_1\), \(M_2\), and \(N_2\), to argue which one is more efficient in
  general: direct 2D convolution or two successive 1D convolutions?
\end{enumerate}

    \textbf{Answer:}

\textbf{i.} \(M_1N_1M_2N_2\) multiplication operations

\textbf{ii.} \(M_1N_1M_2 + M_1N_1N_2\) multiplication operations

\textbf{iii.} Two successive 1D convolutions are more efficient. Direct
2D convolutions have O(\(M_1N_1M_2N_2\)) runtime while two successive 1D
convolutions have O(\(M_1N_1M_2 + M_1N_1N_2\)) runtime.

    Now, we will empirically compare the running time of a separable 2D
convolution and its equivalent two 1D convolutions. The Gaussian kernel,
widely used for blurring images, is one example of a separable filter.
Run the code below to see its effect.

    \begin{tcolorbox}[breakable, size=fbox, boxrule=1pt, pad at break*=1mm,colback=cellbackground, colframe=cellborder]
\prompt{In}{incolor}{177}{\boxspacing}
\begin{Verbatim}[commandchars=\\\{\}]
\PY{c+c1}{\PYZsh{} Load image}
\PY{n}{img} \PY{o}{=} \PY{n}{io}\PY{o}{.}\PY{n}{imread}\PY{p}{(}\PY{l+s+s1}{\PYZsq{}}\PY{l+s+s1}{dog.jpg}\PY{l+s+s1}{\PYZsq{}}\PY{p}{,} \PY{n}{as\PYZus{}gray}\PY{o}{=}\PY{k+kc}{True}\PY{p}{)}

\PY{c+c1}{\PYZsh{} 5x5 Gaussian blur}
\PY{n}{kernel} \PY{o}{=} \PY{n}{np}\PY{o}{.}\PY{n}{array}\PY{p}{(}\PY{p}{[}
    \PY{p}{[}\PY{l+m+mi}{1}\PY{p}{,}\PY{l+m+mi}{4}\PY{p}{,}\PY{l+m+mi}{6}\PY{p}{,}\PY{l+m+mi}{4}\PY{p}{,}\PY{l+m+mi}{1}\PY{p}{]}\PY{p}{,}
    \PY{p}{[}\PY{l+m+mi}{4}\PY{p}{,}\PY{l+m+mi}{16}\PY{p}{,}\PY{l+m+mi}{24}\PY{p}{,}\PY{l+m+mi}{16}\PY{p}{,}\PY{l+m+mi}{4}\PY{p}{]}\PY{p}{,}
    \PY{p}{[}\PY{l+m+mi}{6}\PY{p}{,}\PY{l+m+mi}{24}\PY{p}{,}\PY{l+m+mi}{36}\PY{p}{,}\PY{l+m+mi}{24}\PY{p}{,}\PY{l+m+mi}{6}\PY{p}{]}\PY{p}{,}
    \PY{p}{[}\PY{l+m+mi}{4}\PY{p}{,}\PY{l+m+mi}{16}\PY{p}{,}\PY{l+m+mi}{24}\PY{p}{,}\PY{l+m+mi}{16}\PY{p}{,}\PY{l+m+mi}{4}\PY{p}{]}\PY{p}{,}
    \PY{p}{[}\PY{l+m+mi}{1}\PY{p}{,}\PY{l+m+mi}{4}\PY{p}{,}\PY{l+m+mi}{6}\PY{p}{,}\PY{l+m+mi}{4}\PY{p}{,}\PY{l+m+mi}{1}\PY{p}{]}
\PY{p}{]}\PY{p}{)}

\PY{n}{t0} \PY{o}{=} \PY{n}{time}\PY{p}{(}\PY{p}{)}
\PY{n}{out} \PY{o}{=} \PY{n}{conv\PYZus{}nested}\PY{p}{(}\PY{n}{img}\PY{p}{,} \PY{n}{kernel}\PY{p}{)}
\PY{n}{t1} \PY{o}{=} \PY{n}{time}\PY{p}{(}\PY{p}{)}
\PY{n}{t\PYZus{}normal} \PY{o}{=} \PY{n}{t1} \PY{o}{\PYZhy{}} \PY{n}{t0}

\PY{c+c1}{\PYZsh{} Plot original image}
\PY{n}{plt}\PY{o}{.}\PY{n}{subplot}\PY{p}{(}\PY{l+m+mi}{1}\PY{p}{,}\PY{l+m+mi}{2}\PY{p}{,}\PY{l+m+mi}{1}\PY{p}{)}
\PY{n}{plt}\PY{o}{.}\PY{n}{imshow}\PY{p}{(}\PY{n}{img}\PY{p}{)}
\PY{n}{plt}\PY{o}{.}\PY{n}{title}\PY{p}{(}\PY{l+s+s1}{\PYZsq{}}\PY{l+s+s1}{Original}\PY{l+s+s1}{\PYZsq{}}\PY{p}{)}
\PY{n}{plt}\PY{o}{.}\PY{n}{axis}\PY{p}{(}\PY{l+s+s1}{\PYZsq{}}\PY{l+s+s1}{off}\PY{l+s+s1}{\PYZsq{}}\PY{p}{)}

\PY{c+c1}{\PYZsh{} Plot convolved image}
\PY{n}{plt}\PY{o}{.}\PY{n}{subplot}\PY{p}{(}\PY{l+m+mi}{1}\PY{p}{,}\PY{l+m+mi}{2}\PY{p}{,}\PY{l+m+mi}{2}\PY{p}{)}
\PY{n}{plt}\PY{o}{.}\PY{n}{imshow}\PY{p}{(}\PY{n}{out}\PY{p}{)}
\PY{n}{plt}\PY{o}{.}\PY{n}{title}\PY{p}{(}\PY{l+s+s1}{\PYZsq{}}\PY{l+s+s1}{Blurred}\PY{l+s+s1}{\PYZsq{}}\PY{p}{)}
\PY{n}{plt}\PY{o}{.}\PY{n}{axis}\PY{p}{(}\PY{l+s+s1}{\PYZsq{}}\PY{l+s+s1}{off}\PY{l+s+s1}{\PYZsq{}}\PY{p}{)}

\PY{n}{plt}\PY{o}{.}\PY{n}{show}\PY{p}{(}\PY{p}{)}
\end{Verbatim}
\end{tcolorbox}

    \begin{center}
    \adjustimage{max size={0.9\linewidth}{0.9\paperheight}}{output_38_0.png}
    \end{center}
    { \hspace*{\fill} \\}
    
    In the below code cell, define the two 1D arrays (\texttt{k1} and
\texttt{k2}) whose product is equal to the Gaussian kernel.

    \begin{tcolorbox}[breakable, size=fbox, boxrule=1pt, pad at break*=1mm,colback=cellbackground, colframe=cellborder]
\prompt{In}{incolor}{178}{\boxspacing}
\begin{Verbatim}[commandchars=\\\{\}]
\PY{c+c1}{\PYZsh{} The kernel can be written as outer product of two 1D filters}
\PY{n}{k1} \PY{o}{=} \PY{k+kc}{None}  \PY{c+c1}{\PYZsh{} shape (5, 1)}
\PY{n}{k2} \PY{o}{=} \PY{k+kc}{None}  \PY{c+c1}{\PYZsh{} shape (1, 5)}

\PY{c+c1}{\PYZsh{}\PYZsh{}\PYZsh{} YOUR CODE HERE}
\PY{n}{k1} \PY{o}{=} \PY{n}{np}\PY{o}{.}\PY{n}{array}\PY{p}{(}\PY{p}{[}\PY{l+m+mi}{1}\PY{p}{,}\PY{l+m+mi}{4}\PY{p}{,}\PY{l+m+mi}{6}\PY{p}{,}\PY{l+m+mi}{4}\PY{p}{,}\PY{l+m+mi}{1}\PY{p}{]}\PY{p}{)}\PY{o}{.}\PY{n}{reshape}\PY{p}{(}\PY{l+m+mi}{5}\PY{p}{,}\PY{l+m+mi}{1}\PY{p}{)}
\PY{n}{k2} \PY{o}{=} \PY{n}{np}\PY{o}{.}\PY{n}{array}\PY{p}{(}\PY{p}{[}\PY{l+m+mi}{1}\PY{p}{,}\PY{l+m+mi}{4}\PY{p}{,}\PY{l+m+mi}{6}\PY{p}{,}\PY{l+m+mi}{4}\PY{p}{,}\PY{l+m+mi}{1}\PY{p}{]}\PY{p}{)}\PY{o}{.}\PY{n}{reshape}\PY{p}{(}\PY{l+m+mi}{1}\PY{p}{,}\PY{l+m+mi}{5}\PY{p}{)}
\PY{c+c1}{\PYZsh{}\PYZsh{}\PYZsh{} END YOUR CODE}

\PY{c+c1}{\PYZsh{} Check if kernel is product of k1 and k2}
\PY{k}{if} \PY{o+ow}{not}  \PY{n}{np}\PY{o}{.}\PY{n}{all}\PY{p}{(}\PY{n}{k1} \PY{o}{*} \PY{n}{k2} \PY{o}{==} \PY{n}{kernel}\PY{p}{)}\PY{p}{:}
    \PY{n+nb}{print}\PY{p}{(}\PY{l+s+s1}{\PYZsq{}}\PY{l+s+s1}{k1 * k2 is not equal to kernel}\PY{l+s+s1}{\PYZsq{}}\PY{p}{)}
    
\PY{k}{assert} \PY{n}{k1}\PY{o}{.}\PY{n}{shape} \PY{o}{==} \PY{p}{(}\PY{l+m+mi}{5}\PY{p}{,} \PY{l+m+mi}{1}\PY{p}{)}\PY{p}{,} \PY{l+s+s2}{\PYZdq{}}\PY{l+s+s2}{k1 should have shape (5, 1)}\PY{l+s+s2}{\PYZdq{}}
\PY{k}{assert} \PY{n}{k2}\PY{o}{.}\PY{n}{shape} \PY{o}{==} \PY{p}{(}\PY{l+m+mi}{1}\PY{p}{,} \PY{l+m+mi}{5}\PY{p}{)}\PY{p}{,} \PY{l+s+s2}{\PYZdq{}}\PY{l+s+s2}{k2 should have shape (1, 5)}\PY{l+s+s2}{\PYZdq{}}
\end{Verbatim}
\end{tcolorbox}

    We now apply the two versions of convolution to the same image, and
compare their running time. Note that the outputs of the two
convolutions must be the same.

    \begin{tcolorbox}[breakable, size=fbox, boxrule=1pt, pad at break*=1mm,colback=cellbackground, colframe=cellborder]
\prompt{In}{incolor}{179}{\boxspacing}
\begin{Verbatim}[commandchars=\\\{\}]
\PY{c+c1}{\PYZsh{} Perform two convolutions using k1 and k2}
\PY{n}{t0} \PY{o}{=} \PY{n}{time}\PY{p}{(}\PY{p}{)}
\PY{n}{out\PYZus{}separable} \PY{o}{=} \PY{n}{conv\PYZus{}nested}\PY{p}{(}\PY{n}{img}\PY{p}{,} \PY{n}{k1}\PY{p}{)}
\PY{n}{out\PYZus{}separable} \PY{o}{=} \PY{n}{conv\PYZus{}nested}\PY{p}{(}\PY{n}{out\PYZus{}separable}\PY{p}{,} \PY{n}{k2}\PY{p}{)}
\PY{n}{t1} \PY{o}{=} \PY{n}{time}\PY{p}{(}\PY{p}{)}
\PY{n}{t\PYZus{}separable} \PY{o}{=} \PY{n}{t1} \PY{o}{\PYZhy{}} \PY{n}{t0}

\PY{c+c1}{\PYZsh{} Plot normal convolution image}
\PY{n}{plt}\PY{o}{.}\PY{n}{subplot}\PY{p}{(}\PY{l+m+mi}{1}\PY{p}{,}\PY{l+m+mi}{2}\PY{p}{,}\PY{l+m+mi}{1}\PY{p}{)}
\PY{n}{plt}\PY{o}{.}\PY{n}{imshow}\PY{p}{(}\PY{n}{out}\PY{p}{)}
\PY{n}{plt}\PY{o}{.}\PY{n}{title}\PY{p}{(}\PY{l+s+s1}{\PYZsq{}}\PY{l+s+s1}{Normal convolution}\PY{l+s+s1}{\PYZsq{}}\PY{p}{)}
\PY{n}{plt}\PY{o}{.}\PY{n}{axis}\PY{p}{(}\PY{l+s+s1}{\PYZsq{}}\PY{l+s+s1}{off}\PY{l+s+s1}{\PYZsq{}}\PY{p}{)}

\PY{c+c1}{\PYZsh{} Plot separable convolution image}
\PY{n}{plt}\PY{o}{.}\PY{n}{subplot}\PY{p}{(}\PY{l+m+mi}{1}\PY{p}{,}\PY{l+m+mi}{2}\PY{p}{,}\PY{l+m+mi}{2}\PY{p}{)}
\PY{n}{plt}\PY{o}{.}\PY{n}{imshow}\PY{p}{(}\PY{n}{out\PYZus{}separable}\PY{p}{)}
\PY{n}{plt}\PY{o}{.}\PY{n}{title}\PY{p}{(}\PY{l+s+s1}{\PYZsq{}}\PY{l+s+s1}{Separable convolution}\PY{l+s+s1}{\PYZsq{}}\PY{p}{)}
\PY{n}{plt}\PY{o}{.}\PY{n}{axis}\PY{p}{(}\PY{l+s+s1}{\PYZsq{}}\PY{l+s+s1}{off}\PY{l+s+s1}{\PYZsq{}}\PY{p}{)}

\PY{n}{plt}\PY{o}{.}\PY{n}{show}\PY{p}{(}\PY{p}{)}

\PY{n+nb}{print}\PY{p}{(}\PY{l+s+s2}{\PYZdq{}}\PY{l+s+s2}{Normal convolution: took }\PY{l+s+si}{\PYZpc{}f}\PY{l+s+s2}{ seconds.}\PY{l+s+s2}{\PYZdq{}} \PY{o}{\PYZpc{}} \PY{p}{(}\PY{n}{t\PYZus{}normal}\PY{p}{)}\PY{p}{)}
\PY{n+nb}{print}\PY{p}{(}\PY{l+s+s2}{\PYZdq{}}\PY{l+s+s2}{Separable convolution: took }\PY{l+s+si}{\PYZpc{}f}\PY{l+s+s2}{ seconds.}\PY{l+s+s2}{\PYZdq{}} \PY{o}{\PYZpc{}} \PY{p}{(}\PY{n}{t\PYZus{}separable}\PY{p}{)}\PY{p}{)}
\end{Verbatim}
\end{tcolorbox}

    \begin{center}
    \adjustimage{max size={0.9\linewidth}{0.9\paperheight}}{output_42_0.png}
    \end{center}
    { \hspace*{\fill} \\}
    
    \begin{Verbatim}[commandchars=\\\{\}]
Normal convolution: took 4.073445 seconds.
Separable convolution: took 1.926899 seconds.
    \end{Verbatim}

    \begin{tcolorbox}[breakable, size=fbox, boxrule=1pt, pad at break*=1mm,colback=cellbackground, colframe=cellborder]
\prompt{In}{incolor}{180}{\boxspacing}
\begin{Verbatim}[commandchars=\\\{\}]
\PY{c+c1}{\PYZsh{} Check if the two outputs are equal}
\PY{k}{assert} \PY{n}{np}\PY{o}{.}\PY{n}{max}\PY{p}{(}\PY{n}{out\PYZus{}separable} \PY{o}{\PYZhy{}} \PY{n}{out}\PY{p}{)} \PY{o}{\PYZlt{}} \PY{l+m+mf}{1e\PYZhy{}8}
\end{Verbatim}
\end{tcolorbox}

    \begin{tcolorbox}[breakable, size=fbox, boxrule=1pt, pad at break*=1mm,colback=cellbackground, colframe=cellborder]
\prompt{In}{incolor}{ }{\boxspacing}
\begin{Verbatim}[commandchars=\\\{\}]

\end{Verbatim}
\end{tcolorbox}


    % Add a bibliography block to the postdoc
    
    
    
\end{document}
